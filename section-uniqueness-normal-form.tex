In this section we prove a weak form of confluence: normal form for
all closed terms of $\CTlambda$ of type $\N$ is unique.
By the weak normalization result proved in Section~\ref{section-weak-normalization},
we will deduce: for all closed terms $t$ of $\CTlambda$ of type $\N$, 
there is some $n \in \N$ such that there is a safe reduction $t \safeReduces^* n$, and all normal form of $t$
are equal to $n$.

We define equivalences $\sim_A$ (for each type $A$) for well-typed terms with type $A$.
\begin{definition}
  We define an equivalence $\sim_A$ for \emph{closed} terms of $\CTlambda$
  by induction on the type $A$. 
  \begin{enumerate}
  \item
    $t \sim_N u$ if and only if for all normal $t',u':\N$, if $t \safeReduces^* t'$
    and $u \safeReduces^* u'$ then $t'=u'$.
  \item
    $t \sim_{A\rightarrow B} u$ if and only if for all closed $a_1,a_2:A$ if
    $a_1 \sim_A a_2$ then $t(a_1) \sim_B u(a_2)$.
  \end{enumerate}
  We write $a_1,\ldots,a_n \sim_{A_1,\ldots,A_n} a'_1,\ldots,a'_n$ if $a_i\sim_{A_i} a'_i$ holds
  for all $i\in\{1,\ldots,n\}$. 
  
  We also define $t[\vec{x}] \sim_A u[\vec{x}]$ for \emph{any} terms of $\CTlambda$,
  with $\vec{x}:\vec{A}\vdash t[\vec{x}]:A$ and $\vec{x}:\vec{A}\vdash u[\vec{x}]:A$. 
  \begin{enumerate}
  \item[3.]
    $t[\vec{x}] \sim_A u[\vec{x}]$ if and only if for all closed terms $\vec{a_1}:\vec{A}$ and $\vec{a_2}:\vec{A}$,
    if $\vec{a_1}\sim_{\vec{A}}\vec{a_2}$ then $t[\vec{a_1}] \sim_A u[\vec{a_2}]$. 
  \end{enumerate}  
  
\end{definition}

We will prove that, for all closed terms $t$ of $\CTlambda$ with type $A$, we have $t \sim_A t$.
If $A=\N$ this means that the normal form of all closed terms of $\CTlambda$ of type $\N$ is unique,
which is our goal.

\begin{definition}
  Let $t$ be a term of $\CTlambda$. 
  \begin{enumerate}
  \item
    Assume that $t$ is a closed term with type $N$. 
    We say that $n$ is a value of $t$ if $t \safeReduces^* n$ ($t$ safely reduces to $n$). 
  \item
    We say that $t$ is confluent if $t \sim_A t$.
  \item
    Assume $\vec{x}:\vec{A}\vdash t:\vec{D}\rightarrow\N$. 
    We say a quadraple $(\vec{a};\vec{d};\vec{a'};\vec{d'})$ of vectors consist of closed terms
    is a counterexample to $t$ if
    $\vec{a}\sim_{\vec{A}}\vec{a'}$, $\vec{d}\sim_{\vec{D}}\vec{d'}$, and
    $t[\vec{a}](\vec{d}) \not\sim_N t[\vec{a'}](\vec{d'})$. 
  \end{enumerate}
\end{definition}

By the weak normalization result proved in Section~\ref{section-weak-normalization}, all 
closed term $t$ of $\CTlambda$ of type $\N$ have a value. If $t$ is confluent, then the value is unique.
Hence we have the following. 

\begin{lemma}
  If $t:\N$ is a confluent closed term, then $t$ has a unique normal form.
\end{lemma}

By using the weak normalization result, 
two closed terms $t$ and $u$ equivalent with respect to $\sim_\N$, namely $t \sim_\N u$,
have a common numeral as their unique normal form. 

\begin{lemma}
  Let $t$ and $u$ be closed terms in $\CTlambda$. 
  If $t \sim_\N u$, then there exists $n$ of type $\N$ such that
  (1) $t \safeReduces^* n$, (2) $u \safeReduces^* n$,
  (3) if $t \safeReduces^* t'$ and $t'$ is a normal form, then $t' = n$, and
  (4) if $u \safeReduces^* u'$ and $u'$ is a normal form, then $u' = n$. 
\end{lemma}
\begin{proof}
  By Corollary~\ref{cor:unique_normal_form}, there are numerals $m,n$ such that
  $t\safeReduces^* m$ and $u\safeReduces^* n$.
  Then we obtain (1) and (2), since $m=n$ holds by $t\sim_\N u$.
  If $t \safeReduces^* t'$ and $t'$ is a normal form, then $t'=n$ by $t\sim_\N u$.
  Hence (3) holds. We also have (4) in a similar way. 
\end{proof}

%For confluent closed term $t$ of $\CTlambda$ of type $\N$, "safe" reduction preserves the value. Indeed,
%if $t \reduces u$ then $u$ is a closed term of $\CTlambda$ of type $\N$, therefore $u$ has a value
%$m$, which is also a value of $t$. By confluence of $t$ we conclude that $n=m$.

%If $t$ is as above, with value $n$, and $t \reduces \Succ (u)$, then $n>0$, 
%$u$ is a confluent closed term $t$ of $\CTlambda$ of type $\N$, and the value of $u$ is $n-1$.
%Indeed,if $u \reduces u'$ and $u'$ is normal, then $t \reduces \Succ (u')$, therefore $\Succ (u')=n$ 
%by confluence of $t$, and we conclude that $n>0$ and $u'=n-1$. Thus, the normal form of $u$ is unique.

We have $t \sim_A t$ if and only if there is no counter-example for $t$.
This is shown by the following lemma.
\begin{lemma}
  Let $t$ be a term of $\vec{x}:\vec{A} \vdash t[\vec{x}]:\vec{D}\rightarrow\N$. 
  Let $\vec{a},\vec{d},\vec{a'},\vec{d'}$ be vectors of closed terms.
  Assume that
  $\vec{a}\sim_{\vec{A}}\vec{a'}$ and $\vec{d}\sim_{\vec{D}}\vec{d'}$ implies $t[\vec{a}](\vec{d}) \sim_N t[\vec{a'}](\vec{d'})$. 
  Then $t[\vec{x}]$ is confluent. 
\end{lemma}
\begin{proof}
  Take any $\vec{a},\vec{d},\vec{a'},\vec{d'}$ such that $\vec{a}\sim_{\vec{A}}\vec{a'}$ and $\vec{d}\sim_{\vec{D}}\vec{d'}$.
  Then $t[\vec{a}](\vec{d}) \sim_N t[\vec{a'}](\vec{d'})$ holds by the assumption.
  We have $t[\vec{a}] \sim_{\vec{D}\rightarrow\N} t[\vec{a'}]$ by the definition of $\sim_{\vec{D}\rightarrow\N}$.
  Hence $t[\vec{x}] \sim_{\vec{D}\rightarrow\N} t[\vec{x}]$ holds by the definition. 
\end{proof}

Taking contraposition of this lemma, there exists a counterexample for non-confluent $t$. 
Using this, we prove the following theorem.

\begin{theorem}
  Assume $\Pi:\Gamma\vdash t:A$ (hence $t \in \WTyped$). 
  If $t$ is not confluent,
  then $t \not\in \GTC$, that is, $\Pi$ does not satisfy the global trace condition. 
\end{theorem}
\begin{proof}

\end{proof}

As a corollary from the theorem we will conclude: 
if $t$ satisfies the global trace condition, then there is no such $\sigma$, therefore $t \sim t$, 
as we wished to show.

We prove that for any term $ (t \sim t)$ with counter-example $\vec{a},\vec{d} \sim_0 \vec{b},\vec{e}$
and $\neg (t[\vec{a}](\vec{d}) \sim_0  t[\vec{b}](\vec{e}))$ we can find some immediate
subterm $t'$ and with a counter-example $\vec{a'},\vec{d'} \sim_0 \vec{b'},\vec{e'}$
and $\neg (t'[\vec{a'}](\vec{d'}) \sim_0  t'[\vec{b'}](\vec{e'}))$, \emph{compatible with trace
condition}.


%21:09 20/03/2024
