
\newpage

\section{Appendix: Equivalence Between Cyclic and non-Cyclic System T} 
\label{section-equivalence-cyclic-non-cyclic-T}
\bfColor{red}{(This section is but a draft)}
In this section we prove that the two systems $\systemT$ and $\CTlambda$ 
are each interpretable into the other. 
Both interpretations preserve the reduction relation, applications, $0$ and $\Succ$ and contexts.

In both interpretations 
we neglect the terms of $\CTlambda$ including type variables in their type or in their context,
because these terms have no corresponding in $\systemT$.
\\

We first provide the interpretation from $\systemT$ to $\CTlambda$, which has a short definition.

For any type $T$ we define a term $\Rec:T,(\N,T \rightarrow T),\N\rightarrow T$ such that
$\Rec(a,f,0) = a$ and $\Rec(a,f)(n+1) = f(n,\Rec(a,f,n))$, for all numeral $n \in \Num$.
The definition is $\Rec = \lambda a,f.\rec$ 
with $\rec = \cond (a,\lambda x^{\N}.f(x,\rec(x))) : \N \rightarrow T$.

\begin{proposition}
$\Rec$ is a term of $\CTlambda$.
\end{proposition}

\begin{proof}
\begin{enumerate}
\item
 $\Rec$ is regular by construction.
\item 
The only infinite path of $\Rec$ loops from $\rec$ to $\rec$ infinitely many times, and it includes
an infinitely progressing trace. Here it is: from the first unnamed argument of $\rec$ to 
to the first unnamed argument of $\lambda x^{\N}.f(x,\rec(x))$, then to $x^\N$ 
in the context of $f(x,\rec(x))$, to $x^\N$ in the context of $\rec(x)$,
and eventually to the first unnamed argument of $\rec$ again.
\end{enumerate}
\end{proof}

If we replace each primitive recursion in $\systemT$ with the term $\Rec$ in $\CTlambda$
we define an interpretation from $\systemT$ into $\CTlambda$, 
preserving reductions, applications, $0$ and $\Succ$ and contexts.
\\


The opposite interpretation, from $\CTlambda$ to $\systemT$, it has been defined by proof-theory for the
combinatorial version of circular $\systemT$. For $\CTlambda$, instead we will define an algorithm
taking an infinite term in  $\CTlambda$, described as a finite circular tree, 
and returning a term in $\systemT$ with the required properties. 

First, we need a notion of confluence and of extensional equality 
for  functionals of $\CTlambda$ and of $\systemT$.

We define $t \sim_{\beta,\rec} u$ (a syntactical confluence for $\systemT$) 
if and and only if $t, u \in \systemT$ and $t,u$ have the same type
and for some $v \in \systemT$: $t \reduces_{\beta,\rec} v$ and $u \reduces_{\beta,\rec} v$.

We define an equivalence relation (an extensional equality for $\systemT$) 
$\sim_{\systemT}$ on $\systemT$, by induction on the type. 


\begin{definition}[An extensional equality on $\systemT$]
Assume $t,u \in \systemT$.
\begin{enumerate}
\item
If $t,u:\N$ and $t,u$ are closed then we set: 
$(t \sim_{\systemT} u) \Leftrightarrow  (t \sim_{\beta,\rec} u)$.
\item
If $t,u:A,\vec{A}\rightarrow\N$ and $t,u$ are closed then we set: 
$(t \sim_{\systemT} u) \Leftrightarrow  
\forall a \in \systemT. (a:A), (\mbox{$a$ closed}) \Rightarrow (t(a) \sim_{\systemT} u(a))$.
\item
If $t,u:\vec{A}\rightarrow\N$ and $\FV(t), \FV(u) \subseteq \vec{x}$ then we set:
$$
(t \sim_{\systemT} u) 
\Leftrightarrow  
\forall \vec{a} \in \systemT. 
(\vec{a}:\vec{A}), (\mbox{$\vec{a}$ closed})  
\Rightarrow 
(t[\vec{a}/\vec{x}] \sim_{\systemT} u[\vec{a}/\vec{x}])
$$
\end{enumerate}
\end{definition}

Then we can consider $\systemT$ as a structure $(\systemT/\sim_{\systemT}, 0, \Succ, \ap)$
with natural numbers, functionals and extensional equality. 
In the same way we define an equivalence relation on terms of $\CTlambda$ which denote functionals.
We only consider terms whose type and context only include the atomic type $\N$, and no type variables. 


\begin{definition}[An extensional equality on $\CTlambda$]
Assume $t,u \in \CTlambda$.
\begin{enumerate}
\item
If $t,u:\N$ and $t,u$ are closed then we set: 
$(t \sim_{\CTlambda} u) \Leftrightarrow  (t \sim_{\CTlambda} u)$.
\item
If $t,u:A,\vec{A}\rightarrow\N$ and $t,u$ are closed then we set: 
$(t \sim_{\CTlambda} u) \Leftrightarrow  
\forall a \in \CTlambda. (a:A), (\mbox{$a$ closed}) \Rightarrow (t(a) \sim_{\CTlambda} u(a))$.
\item
If $t,u:\vec{A}\rightarrow\N$ and $\FV(t), \FV(u) \subseteq \vec{x}$ then we set:
$(t \sim_{\CTlambda} u) 
\Leftrightarrow  
\forall \vec{a} \in \CTlambda. 
(\vec{a}:\vec{A}), (\mbox{$\vec{a}$ closed})  \Rightarrow (t[\vec{a}/\vec{x}] \sim_{\CTlambda} u[\vec{a}/\vec{x}])$.
\end{enumerate}
\end{definition}

Our goal is now to prove that there is a partial isomorphism from the types of $\N$-functionals in 
$(\CTlambda/ \!\! \sim_{\CTlambda}, 0, \Succ, \ap)$ to the entire
$(\systemT/ \!\! \sim_{\systemT}, 0, \Succ, \ap)$.
We ignore types of $\CTlambda$ including type variables, they have no corresponding in $\systemT$.

For each $\N$-functional type $T$ we have to define a map $\phi_T$ from terms of type $T$ of $\CTlambda$
to terms of type $T$ of $\systemT$. We abbreviate $\phi_T$ with $\phi$ and we require:

\begin{enumerate}
\item
If $t \sim_{\CTlambda} u$ \ \ \ \ \ \ \ \ \ \ \ \ \ then $\phi(t) \sim_{\systemT} \phi(u)$

\item
If $t: A \rightarrow B$, $u:A$ \ \ \ then $\phi(t(u)) \sim_{\systemT} \phi(t)(\phi(u))$

\item
$\phi(0) \sim_{\systemT} 0$ and
$\phi(\Succ(t)) \sim_{\systemT} \Succ(\phi(t))$

\end{enumerate}

We suppose be fixed a cyclic $\lambda$-term $t \in \CTlambda$, $t : T$,
 and we use several ingredients. 

%14:47 11/06/2024

\begin{enumerate}
\item
We suppose be given a map $\trunk_t:\N \rightarrow T$, such that $\trunk_t(n)$ is
the unfolding of $t$ with all subterms $u:U$ in the level number $n$ replaced by a dummy term $0_U$.
\item
For each pair of subterms $u,v$ of $t$ and each path $\pi$ from $u$ to $v$. 
we suppose be given a one-to-many trace
relation $R_\pi$ between the indexes of $\N$-arguments of $u$ and of the $\N$-arguments of $v$. 
We close the set of relations $R_\pi$ by composition 
and we obtain a finite set (of exponential size in $t$).
\end{enumerate}

By the global trace condition for each infinite composition of $R_\pi$ there is some infinitely progressing
trace. This means that there is some index $i$ in the domain of $R_\pi$ such that for some $n \in \N$
we have $R^n_\pi(i,i)$ and the variable $i$ progresses.


We consider any assignment $t' \equiv t[\vec{n},\vec{x}](\vec{m},\vec{y})$ of the arguments of $t$.
The idea is to find map $\phi$ such that for all $p \ge \vec{n}, \vec{n}$ we have
$\trunk_{t'}(m)$ stationary for all $m \ge \phi(p)$, therefore $t' = \trunk_{t'}(\phi(p))$.

$\phi_c(p)$ is the map computing the maximum number of nodes for an Erdos tree in $c$
colors with height $\le p$ and branching $\le c$ (there is at most one child per color).
Whenever an Erdos tree has at least a branch of length $cp+1$ and $c$ colors, 
then we have a monotonically colored sequence
of length $cp+1$, therefore at least one homogeneous set of length $p+1$. If we take any composition of 
$\phi_c(p)+1$ times some relations $R_\pi$, there is some homogeneous set of $(p+1)$ elements
decorated with a single $R_\pi$, and for some $i$ some variable starting from some value $\le p$ 
and decreasing $p$ times. 
This means that each branch terminates and the whole computation of $\trunk_{t'}(\phi(p))$ terminates.

For a $c$-color tree, the value of $\phi_c(p)$ is $1+c+c^2+\ldots+c^{p-1} = (c^{p}-1)/(c-1)$.
Suppose $f:\N \rightarrow \N$ is some weakly increasing map. We want to prove that
there is some Erdos tree including some branch including some $f$-homogeneous set: some
homogeneous set with first point $h$ followed by $f(h)$ more points. We
want to define a functional in $\systemT$ such that all Erdos trees with $\ge F(f)$ nodes
include some $f$-homogeneous set.
We define $x_1 = 0$, $x_{h+1} = 1+ (c^{f(x_h)}-1)/(c-1)$ and $F(f) = x_{c+1}$.
%?????????????????????
%22:31 10/06/2024

\ldots\ldots\ldots

%We require the following property of $\systemT$: we can define terms of $\systemT$ by $n$
%simultaneous lexicographic inductions. If $\vec{n} \in \N^m$ are numeral, we define 
%$\vec{n} \lexicographic{} \vec{m}$ if and only if $\vec{n} \not = \vec{m}$ and for the first $i \in [1,m]$
%such that $n_i \not = m_i$ we have $n_i +1 = m_i$. 
%If $T=\vec{A} \rightarrow \N$ is a functional type we define
%$0_T = \lambda \vec{x}:\vec{A}.0$. If $f:T$ we define $f \restr  \vec{x}$
%as the restriction of $f_i$ to the set of $\vec{y} \lexicographic{} \vec{x}$, 
%extended by the dummy value $0_T$:
%\begin{center}
% $(f \restr  \vec{x})(\vec{y}) = f(\vec{y})$ if $\vec{y} \lexicographic{} \vec{x}$ 
%\ \ \ 
%and 
%\ \ \ 
%$(f \restr  \vec{x})(\vec{y}) = 0_T$ otherwise
%\end{center}

%
%\begin{proposition}[Simultaneous lexicographic induction in $\systemT$]
%Assume $\vec{x}:\N^m$,
%and that we have any equation list in $\systemT$, in the meta-variables $f_1, \ldots, f_n$:
%$$
%f_i(\vec{x}) 
%\ \ \ 
%\sim_{\systemT} 
%\ \ \ 
%F_i(\vec{x}, f_1 \restr  \vec{x}, \ldots, f_n \restr  \vec{x})
%\ \ \ 
% : 
%\ \ \ 
%T_i
%$$
%This  equation list has solutions 
%$
%f_1:\N^m \rightarrow T_1, 
%\ldots, 
%f_n:\N^m \rightarrow T_n
%$ 
%in $\systemT$ and we can compute them. 
%\end{proposition}
%
%
%Assume $t:T$ is a cyclic term, represented as a cyclic tree with node $t_1$, \ldots, $t_n$.
%We translate it to a term of $\systemT$, obtained by solving an equation list whose meta-variables
%are the nodes the cyclic tree, with $\vec{x} = \vec{y},c$,
%and $\vec{y}$ the union of the $\N$-arguments of each node,
%and $c$ a variable used as a counter. The counter 
%$c$ starts from $n$, the number of nodes, and decreases
%of $1$ unit in each recursive call. 
%Within $n$ recursive calls, one or more value of $\vec{y}$ decreases by $1$,
%while all other values stay the same. In this case the varabile $c$ is reset to $n$: $c$ is the only
%variable which can increase during computation.
%
%
%%
%%This is a first draft about how to do it.
%%
%%%%%%%%%%%%%%%
%% % TO BE IMPROVED
%%%%%%%%%%%%%%%
%%
%%\begin{enumerate}
%%
%%\item
%%We first move all nodes to a context with the same number on type $\N$ variables, 
%%by adding dummy variables and dummy arguments.
%%This operation preserves regularity and global trace condition.
%%Now $t_1$, \ldots, $t_n$ all have context $\Gamma$ and type $A$.
%%
%%\item
%%We merge all buds into the same term, defined by some $u$ such that $u(i)=t_i$, for $i=1, \ldots, n$,
%%and $u(i)=$ some dummy term of type $A$ otherwise. We replace each $t_i$ with $u(i)$, 
%%for $i=1, \ldots, n$.
%%This operation preserves regularity and global trace condition.
%%Now we have $n$ buds, all are the same $u$ with context $\Gamma$ and type $\N \rightarrow A$.
%%Each bud $b$ defines a partial bijection between the occurrence of $\N$ in its context and type
%%$\Gamma \vdash \N \rightarrow A$, and the occurrences of $\N$ in the context and type
%%$\Gamma \vdash \N \rightarrow A$ of its companion. 
%%We extend this partial bijection to any total bijection $\tau$, depending on the but $b$.
%%
%%\item
%%We close the partial bijections defined by each bud by composition. The number of partial 
%%bijections can grow in an exponential  way.
%%
%%\item
%%Assume we have $m$ occurrences of $\N$ inside the context and type 
%%$\Gamma \vdash \N \rightarrow A$ of $u$.
%%We fix a permutation $\sigma:\{1, \ldots, m\}$ 
%%and we label them by variables $x_1, \ldots, x_n$ of $\systemT$,
%%with $x_i$ label of the argument with type $\N$ and number $i$.
%%We will define a translation $t^\sigma \in \systemT$ of  $t \in \CTlambda$.
%%
%%\item
%%All traces move from $u$ to any of the occurrences of $u$ inside $u$. 
%%Some traces of some $\N$
%%in $\Gamma \vdash \N \rightarrow A$ disappear, some other are moved to some other $\N$,
%%in an injective way. Two traces never merge.
%%We label each trace in the bud $u$ with the name $x_i$ of the corresponding trace, if any.
%%All those corresponding to no trace are labeled at random using the remaining variable names.
%%
%%At least one trace progresses, otherwise by repeating infinitely many times this step we would get a
%%path with no progressing trace. The same is true for any combination of one or more movements
%%from $u$ to $u$. 
%%
%%%After $m$ movements to any $u$ inside $u$, 
%%%each of the $m$ traces either disappeared or cycles. After $m!$ steps, all
%%%cycles are back to their original point. 
%%%
%%%All traces are now restarted or move from one $\N$ to the same $\N$, with or without progression.
%%
%%\item
%%At least one trace $x_i$ progresses and it is not erased by any other trace. Otherwise we could follow a path
%%in which each progress is erased in some new step, and so there is no infinite progressing trace.
%%We use this trace as the main variable $x_i$ of the recursion. In all steps, either $x_i$ is constant or decreases,
%%and in at least one case it decreases. In all cases in which $x_i$ decrease we use primitive
%%recursion on $x_i$ in $\systemT$, as main variable. 
%%In all other case, $x_i$ is not removed, therefore it stays the same. 
%%We isolate the main variable $x_j$ of the recursion for these steps, it is progressing therefore $j \not = i$.
%%We use primitive recursion on $x_j$: this is the second variable of primitive recursion. 
%%We continue in this way and we define a primitive recursion in $\systemT$, with pairwise distinct 
%%indexes $x_{i_1} = x_i$, $x_{i_2} = x_j$, \ldots, $x_{i_k}$ for some $k \ge 1$. We extend 
%%$x_{i_1}, \ldots, x_{i_k}$ to $x_{i_1}, \ldots, x_{i_n}$ in a random way: we defined in this way a
%%permutation $\sigma$ on $\{1, \ldots, m\}$ by $\sigma(j) = i_j$ for $j \in \{1, \ldots, m\}$
%%We define in this way a closed primitive recursive term 
%%$\lambda \vec{x}.t^\sigma \in \systemT$. Each bud $u$
%%defining a permutation $\tau$ is replaced by $\exch_{\tau}(f)$.
%%The term $\exch_\tau \in \systemT$ applies the permutation $\tau$ to the arguments of $f$,
%%and during the recursive call $f$ is replaced by $\lambda \vec{x}.u^\sigma$.
%%\end{enumerate}
%%
%%We claim that the infinite term $t^\sigma \in \systemT$ 
%%is equivalent to the cyclic recursive term $t \in \CTlambda$ we started from.
