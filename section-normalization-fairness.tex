\\
\emph{(Here we should insert our notes for the proof of strong normalization for safe reductions)}
\ldots\ldots\ldots\ldots\ldots\ldots\ldots\ldots\ldots\ldots
\\

There are terms requiring infinite reductions to normalize, but there are reductions normalizing in the limit.
We conjecture that we can characterize reductions normalizing in the limit using the notion of "fair" reductions.

\begin{definition}
\begin{enumerate}
\item
We say that a node is in the $k,\cond$-level if it has less than $k$ nodes $\cond$ in its branch.
\item
We say that an infinite reduction sequence is is \emph{fair} for nodes of $k,\cond$-level in the following case:
for all $i\in \N$, all $k>0$, there is some  $j \ge i$ such that at step $j$ either all nodes in the 
$k,\cond$-level are normal, or one of them is reduced at step $j$.
\end{enumerate}
\end{definition}

\emph{Conjecture 1}. For all $k \in \N$, an infinite reduction sequence $\sigma$ 
eventually stop reducing nodes in the $k,\cond$-level.

Conjecture 1 should be proved by adapting the proof of strong normalization for safe reductions to a
termination result on the first $k$-levels for all reductions, and assuming the same result for $k-1$.

\emph{Conjecture 2}. An infinite reduction sequence $\sigma$ 
is normalizing in the limit if and only if for all $k \in \N$,
$k > 0$, $\sigma$ is fair for the task of reducing nodes in the $k,\cond$-level.

Conjecture 2 should follow in few steps from conjecture Conjecture 1.

