% lmcs, 2018.5.2, 2019.9.9, 2020.2.10



\newif\ifdraft \draftfalse
\drafttrue % put % for lipics


\long\def\Stefano#1{{{\color{red}{SB: #1}}}}
\long\def\Makoto#1{{{\color{blue}{MT: #1}}}}
\long\def\Daisuke#1{{{\color{green}{DK: #1}}}}

\ifdraft

\documentclass{article}
\usepackage{mystyle}
\A4page

\else

\documentclass{lmcs}
\usepackage{hyperref}
%\usepackage{amsmath}heoremstyle{plain}\newtheorem{satz}[thm]{Satz} %\crefname{satz}{Satz}{\Succ    atze}

\fi

\usepackage{mymath,proof,latexsym}
\usepackage{xcolor}
\usepackage[pdftex,outline]{contour}

%%%%%%%%%%%%%%%%%%%%%%%%%%%%%%%%%
% AGGIUNGO TEOREMA LEMMA PROPOSIZIONE COROLLARIO PROVA % 
%%%%%%%%%%%%%%%%%%%%%%%%%%%%%%%%%

\newtheorem{theorem}{Theorem}[section]
\newtheorem{lemma}[theorem]{Lemma}
\newtheorem{proposition}[theorem]{Proposition}
\newtheorem{corollary}[theorem]{Corollary}
\newtheorem{definition}[theorem]{Definition}

\newenvironment{proof}[1][Proof]{\begin{trivlist}
\item[\hskip \labelsep {\bfseries #1}]}{\end{trivlist}}
\newenvironment{example}[1][Example]{\begin{trivlist}
\item[\hskip \labelsep {\bfseries #1}]}{\end{trivlist}}
\newenvironment{remark}[1][Remark]{\begin{trivlist}
\item[\hskip \labelsep {\bfseries #1}]}{\end{trivlist}}


\begin{document}
\sloppy 
\hbadness=10000
\vbadness=10000

% cut note 2

\def\Expand{{\rm expand}}

% cut elim

\def\RCLKIDomega{{{\bf RCLKID}^\omega}}
\def\Ren{{\rm Ren}}
\def\Terms{{\rm Terms}}
\def\IE{{\rm IE}}
\def\Cut{{\rm Cut}}
\def\Rcut{{\rm Rcut}}

% cmcs

\def\Co{{\rm co}}
\def\Axiom{{\rm Axiom}}
\def\Wk{{\rm Wk}}
\def\Cut{{\rm Cut}}
\def\BS{{\rm BS}}
\def\Bit{{\rm Bit}}
\def\LK{{\rm LK}}
\def\Head{{\rm head}}
\def\Tail{{\rm tail}}

% intuitionistic

\def\Univ{{\rm Univ}}
\def\Left{{\rm Left}}
\def\Right{{\rm Right}}
\def\Set{{\rm set}}
\def\ES{{\rm ES}}
\def\PC{{\rm PC}}
\def\Fin{{\rm Fin}}
\def\CLJIDomega{{{\bf CLJID}^\omega}}
\def\CLJIDHA{{{\bf CLJID}^\omega+{\bf HA}}}
\def\LJIDHA{{{\bf LJID}+{\bf HA}}}
\def\LJID{{\bf LJID}}
\def\OP{{\rm OP}}
\def\Div{{\rm Div}}
\def\RP{{\rm RP}}
\def\MS{{\rm MS}}
\def\ET{{\rm ET}}
\def\LiftedTree{{\rm LiftedTree}}
\def\First{{\rm first}}
\def\Er{{\rm Er}}
\def\ER{{\rm ER}}
\def\Insert{{\rm insert}}
\def\ET{{\rm ET}}
\def\SInd{{\rm SInd}}
\def\Trans{{\rm Trans}}
\def\HA{{\bf HA}}
\def\DM{{\rm DM}}
\def\Ext{{\rm ext}}
\def\KB{{\rm KB}}
\def\DT{{\rm DT}}
\def\DS{{\rm DS}}
\def\Monoseq{{\rm Monoseq}}
\def\Eq{{\rm seq}}
\def\Last{{\rm last}}

% equivalence cyclic proofs

%\def\Uor{\displaystyle \mathop {\widetilde\bigvee}}
\def\Uor{\biguplus}
\def\CLKIDomega{{{\bf CLKID}^\omega}}
\def\CLKIDPA{{{\bf CLKID}^\omega+{\bf PA}}}
\def\LKIDPA{{{\bf LKID}+{\bf PA}}}
\def\PA{{\bf PA}}

% simple equiv note

\def\LKIDN{{\bf LKIDN}}
\def\CLKIDN{{\bf CLKIDN}}

% equiv note

\def\Choose{{\rm Choose}}
\def\Asym{{\rm Asym}}
\def\Peano{{\bf Peano}}
\def\Code#1{{\lceil{#1}\rceil}}
\def\Ineq{{\rm Ineq}}
\def\MonoPath{{\rm MonoPath}}
\def\BoundedPath{{\rm BoundedPath}}
\def\InfPath{{\rm InfPath}}
\def\InfMonoPath{{\rm InfMonoPath}}
\def\HomSeq{{\rm HomSeq}}
\def\InfHomSeq{{\rm InfHomSeq}}

\def\ErdosTree{{\rm ErdosTree}}
\def\KB{{\rm KB}}
\def\MonoList{{\rm MonoList}}
%\def\Seq{{\rm Seq}}
\def\Univ{{\rm univ}}
\def\Infinite{{\rm Infinite}}
\def\S{{\bf S}}
%\def \N {{N}} %\def\N{{\bf N}} %
\def\Ind{{\rm Ind}}
\def\LKID{{\bf LKID}}
\def\PA{{\bf PA}}
\def\LKIDExt{{\bf LKIDExt}}
\def\CLKID{{\bf CLKID^\omega}}

% LICS

\def\Over#1#2{\deduce{#2}{#1}}
\def\List{{\rm List}}
\def\ListX{{\rm ListX}}
\def\ListO{{\rm ListO}}
\def\ListE{{\rm ListE}}
\def\Nl{{\rm nl}}

% CSLID

\def\SLLID{{\rm SLID}}
\def\SLID{{\rm SLID}}
\def\CSLID{{\rm CSLID}}
\def\CSLIDomega{{{\rm CSLID}\omega}}
\def\Eclass{{\rm Eclass}}
\def\Eq{{\rm Eq}}
\def\Deq{{\rm Deq}}
\def\Satom{{\rm Satom}}
\def\Cells{{\rm Cells}}
\def\Roots{{\rm Roots}}
\def\Elim{{\rm Elim}}
% \def\Case#1{{\rm Case-$#1$}}
\def\Case{{\rm Case}}
\def\Unfold{{\rm Unfold}}
\def\Pred{{\rm Pred}}
\def\Start{{\rm Start}}
\def\Unsat{{\rm Unsat}}
\def\Split{{\rm Split}}
\def\Underscore{\underline{\phantom{x}}}
\def\Rootcell{{\rm rootcell}}
\def\Rootshape{{\rm rootshape}}
\def\Jointleaf{{\rm jointleaf}}
\def\Jointleafimplicit{{\rm jointleafimplicit}}
\def\Jointnode{{\rm jointnode}}
\def\Directjoint{{\rm directjoint}}

% cyclicnote2

\def\Range{{\rm Range}}
\def\Xstart{X_{{\rm start}}}
\def\kstart{k_{{\rm start}}}
\def\Leaves{{\rm Leaves}}
\def\PureDist{{\rm PureDist}}
\def\Pure{{\rm Pure}}
\def\Identity{{\rm Identity}}
\def\Subst{{\rm Subst}}

% weak establish

\def\Connected{{\rm Connected}}
\def\Eststablished{{\rm Eststablished}}
\def\Valued{{\rm Valued}}

% monadic new

\def\SLRDbtw{{\rm SLRD}_{btw}}
\def\BTW{{{\rm BTW}}}
\def\Loc{{{\rm Loc}}}
\def\Ne{{\ne}}
\def\Nil{{{\rm nil}}}
\def\eqDef{=_{{\rm def}}}
\def\Inf#1{{\infty_{#1}}}
\def\Stores{{\rm Stores}}
\def\SVars{{{\rm SVars}}}
\def\Val{{{\rm Val}}}
\def\MSO{{{\rm MSO}}}
\def\Sep{{{\rm Sep}}}
\def\Septwo{{{\rm SLMI}}}
\def\SLMI{{{\rm SLMI}}}
\def\Sepinf{{\rm Sep}\infty}
\def\THeaps{{\rm THeaps}}
\def\Root{{\rm Root}}
\def\TG{{\rm TGraph}}

% monadic

\def\Nil{{{\rm nil}}}
\def\eqDef{=_{{\rm def}}}
\def\Inf#1{{\infty_{#1}}}
\def\Stores{{\rm Stores}}
\def\SVars{{{\rm SVars}}}
\def\Val{{{\rm Val}}}
\def\MSO{{{\rm MSO}}}
\def\Sep{{{\rm sep}}}
\def\THeaps{{\rm THeaps}}
\def\Root{{\rm Root}}
\def\TG{{\rm TGraph}}

\def\Tilde{\widetilde}
\def\Bar{\overline}
\def\Lequiv{\Longleftrightarrow}
\def\Lto{\Longrightarrow}
\def\Lfrom{\Longleftarrow}
\def\Norm{{\rm Norm}}
\def\Noshare{{\rm Noshare}}
\def\Roots{{\rm Roots}}
\def\Forest{{\rm Forest}}
\def\Var{{\rm Var}}
\def\Range{{\rm Range}}
\def\Cell{{\rm Cell}}
%\def\Tree{{\rm Tree}}
\def\Switch{{\rm Switch}}
\def\All{{\rm All}}
\def\To{\leadsto}
\def\tree{{\rm tree}}
\def\Paths{{\rm Paths}}
\def\Finite{{\rm Finite}}
\def\Const{{\rm Const}}
\def\Leaf{{\rm Leaf}}
\def\LeastElem{{\rm LeastElem}}
\def\LeastIndex{{\rm LeastIndex}}
\def\WSnS{{{\rm WSnS}}}
\def\Expand{{{\rm Expand}}}

% previous

\long\def\J#1{} % skip
\def\T#1{\hbox{\color{green}{$\clubsuit #1$}}}
\def\W#1{\hbox{\color{Orange}{$\spadesuit #1$}}}

\def\Node{{{\rm Node}}}
\def\LL{{{\rm LL}}}
\def\DSN{{{\rm DSN}}}
\def\DCL{{{\rm DCL}}}
\def\LS{{{\rm LS}}}
\def\Ls{{{\rm ls}}}

\def\FPV{{{\rm FPV}}}
\def\Lfp{{{\rm lfp}}}
\def\IsHeap{{{\rm IsHeap}}}

\def\Equiv{\quad \equiv\quad }
\def\Null{{{\rm null}}}
\def\Emp{{{\rm emp}}}
\def\If{{{\rm if\ }}}
\def\Then{{{\rm \ then\ }}}
\def\Else{{{\rm \ else\ }}}
\def\While{{{\rm while\ }}}
\def\Do{{{\rm \ do\ }}}
\def\Cons{{{\rm cons}}}
\def\Dispose{{{\rm dispose}}}
\def\Vars{{{\rm Vars}}}
\def\Locs{{{\rm Locs}}}
\def\States{{{\rm States}}}
\def\Heaps{{{\rm Heaps}}}
\def\FV{{{\rm FV}}}
\def\True{{{\rm true}}}
\def\False{{{\rm false}}}
\def\Dom{{{\rm Dom}}}
\def\Abort{{{\rm abort}}}
\def\New{{{\rm New}}}
\def\W{{{\rm W}}}
\def\Pair{{{\rm Pair}}}
\def\Lh{{{\rm Lh}}}
\def\lh{{{\rm lh}}}
\def\Elem{{{\rm Elem}}}
\def\EEval{{{\rm EEval}}}
% \def\BEval{{{\rm BEval}}} % not used 
\def\PEval{{{\rm PEval}}}
\def\HEval{{{\rm HEval}}}
\def\EVal{{{\rm Eval}}}
\def\Domain{{{\rm Domain}}}
\def\Exec{{{\rm Exec}}}
\def\Store{{{\rm Store}}}
\def\Heap{{{\rm Heap}}}
\def\Storecode{{{\rm Storecode}}}
\def\Heapcode{{{\rm Heapcode}}}
\def\Lesslh{{{\rm Lesslh}}}
\def\Addseq{{{\rm Addseq}}}
\def\Separate{{{\rm Separate}}}
\def\Result{{{\rm Result}}}
\def\Lookup{{{\rm Lookup}}}
\def\ChangeStore{{{\rm ChangeStore}}}
\def\ChangeHeap{{{\rm ChangeHeap}}}
\def\Wand{\mathbin{\hbox{\hbox{---}$*$}}}
%\def\Eval#1{[\kern -1pt[{#1}]\kern -1pt]}
\def\Eval#1{\llbracket{#1}\rrbracket}
\def\Vec{\overrightarrow}

\def\Tilde{\widetilde}
\def\Break{\hfil\break\hbox{}}

%Macro for the work on Cyclic System T with lambda
%22/03/2024
\newcommand{\ap}                { {\tt ap} }
\newcommand{\cond}            { {\tt cond} }
\newcommand{\reduces}       { \mathbin{\mapsto} }
\newcommand{\CTlambda}    { { {\tt CT} \mbox{-} \lambda} }
%\newcommand{\nameIn}       { {\tt name\mbox{-}in} }
\newcommand{\etaRule}       { {\eta\mbox{-rule}} }
\newcommand{\ins}               { {\tt ins} }
\newcommand{\struct}          { {\tt struct} }
\newcommand{\Iter}              { {\tt It} }
\newcommand{\iter}               { {\tt it} }
\newcommand{\Interval}       { {\tt In} }
\newcommand{\interval}        { {\tt in} }
\newcommand{\Succ}             { {\tt S} }
\newcommand{\nil}                { {\tt nil} }
\newcommand{\cons}             { {\tt cons} }
\newcommand{\outline}  [1]   { {\contour{black}{\textcolor{white}{#1}}}  }
\newcommand{\systemT}       { {\outline{\tt T}} }
\newcommand{\LAMBDA}       { {\outline{$\Lambda$}} }
\newcommand{\Rec}               { {\tt Rec} }
\newcommand{\rec}                { {\tt rec} }
\newcommand{\exch}             { {\tt exch} }
\newcommand{\composed}     { {\mbox{\tiny $\circ$}} }
\newcommand{\WTyped}        { {\tt WTyped} }
\newcommand{\safe}              { {\hbox{\scriptsize \tt safe}} }
\newcommand{\GlobalTraceCondition}  { {\tt GlTrCon} }
\newcommand{\Sum}              { {\tt Sum} }
\newcommand{\safeReduces} {\mathbin{\reduces_\safe}}
\newcommand{\safeReducesAst} {\mathbin{\reduces_\safe^*}}
\newcommand{\var}                { {\tt var} }
\newcommand{\Graph}            { {\tt Graph} }
\newcommand{\bfColor} [2]     {{\bf \color{#1}{#2}}}
\newcommand{\weak}             {{\tt weak}}

%colors from the web site: https://latexcolor.com/
\definecolor{amber}{rgb}{1.0, 0.75, 0.0}
\definecolor{oldgold}{rgb}{0.81, 0.71, 0.23}

\newcommand{\SubTerm}               {{\tt SubT}}
\newcommand{\Tree}                      {{\tt Tree}}
\newcommand{\N}                           {{N}}
\newcommand{\Num}                      {{\tt Num}}
\newcommand{\Nat}                        {{\outline{\tt N}}}
\newcommand{\GTC}                       {{\tt GTC}}
\newcommand{\Reg}                       {{\tt Reg}}
\newcommand{\Type}                      {{\tt Type}}
\newcommand{\conc}                      {{\star}}
\newcommand{\typingRule}            {{\tt r}}
%\newcommand{\apvar}                     {{\ap_{\mbox{\tiny var}}}}
%\newcommand{\apnotvar}               {{\ap_{\neg\mbox{\tiny var}}}}
\newcommand{\apvar}                    {{{\tt ap}_{\mbox{\tiny $v$}}}}
\newcommand{\apnotvar}               {{{\tt ap}_{\mbox{\tiny $\neg v$}}}}
\newcommand{\subseteqsim}         {{\raisebox{2pt}{$\subset$}_{\!\!\!\!\!\sim}}}
\newcommand{\supseteqsim}         {{\raisebox{2pt}{$\supset$}_{\!\!\!\!\!\sim}}}
\newcommand{\Ctxt}                       {{\tt Ctxt}}
\newcommand{\Seq}                        {{\tt Seq}}
\newcommand{\Stat}                       {{\tt Stat}}
\newcommand{\Rule}                       {{\tt Rule}}
\newcommand{\restr}                      {{\lceil}}
\newcommand{\setT}                       {{\mathcal T}}
\newcommand{\base}                      {{\tt base}}
\newcommand{\inductive}               {{\tt ind}}
\newcommand{\universe}     [1]       {{|#1|}}
\newcommand{\Label}                     {{\tt Lbl}}
\newcommand{\quotationMarks} [1]     {{\textquotedblleft#1\textquotedblright}}
\newcommand{\lexicographic} [1]   {{ <^{#1}_{\mbox{\tiny lex}}} }
\newcommand{\trunk}                     {{\tt trunk}}






\ifdraft

\title{CT-$\lambda$, a cyclic simply typed $\lambda$-calculus with fixed points}

\author{Stefano Berardi }
\date{}

\else

\title[Equivalence]
{
???
}

\author[S. Berardi]{Stefano Berardi}
\address{Universit\`{a} di Torino,
Torino, Italy}
\email{stefano@unito.it}


%% mandatory lists of keywords 
\keywords{
proof theory
inductive definitions
Brotherston-Simpson conjecture
cyclic proofs
}
\fi

\maketitle

\begin{abstract}
We explore the possibility of defining a circular syntax
directly for $\lambda$-abstraction and simple types, instead of interpreting
 $\lambda$-abstraction through combinators first, then inserting a  circular syntax afterwards.
We introduce our circular syntax as a fixed point operator defined by cases.

We prove the expected results for this circular simply typed $\lambda$-calculus, which we call $\CTlambda$: 
\emph{(i)} every closed term of type $\N$ reduces to some numeral;
\emph{(ii)}  strong normalization for all reductions sequences outside all fixed points; 
\emph{(iii)}  strong normalization for ``fair'' reductions; \emph{(iv)} Church-Rosser property;
and \emph{(i)} the equivalence between circular syntax and ordinary syntax 
for G\"{o}del system $\systemT$. 

Our motivation is that terms of $\CTlambda$ 
are much shorter than the equivalent terms written in $\systemT$.
Beside, the circular syntax of $\CTlambda$, using binders instead of combinators, 
could be more familiar for researchers working in the field of Type Theory.

\end{abstract}

\iffalse
key words: 
proof theory,
inductive definitions,
Brotherston-Simpson conjecture,
cyclic proofs,
Martin-Lof's system of inductive definitions,
infinite Ramsey theorem
Podelski-Rybalchenko termination theorem
size-change termination theorem
\fi



\section{Introduction}
We will introduce $\LAMBDA$, a set of infinite $\lambda$-terms whose types are: the atomic type $\N$, 
possibly type variables $\alpha, \beta, \ldots$, and all types $A \rightarrow B$ for any types $A$, $B$. 
The terms of $\LAMBDA$  are possibly infinite trees representing expressions defined with 
$0$,$\Succ $,$\ap$ (application), 
typed variables,  $\lambda$ (the binder for defining maps), 
and $\cond$, a fixed point binder operator for the arithmetic conditional (test on zero). 
If we have no type variables, the the trees in $\LAMBDA$ represent partial functionals on $\N$, 
provided we add reduction rules.

In this paper will consider the subsets $\CTlambda \subset \GTC, \Reg, \WTyped \subseteq \LAMBDA$,
$\WTyped$ is the set of well-typed terms of $\LAMBDA$. 
$\GTC \subseteq \WTyped$ is the subset of well-typed circular 
$\lambda$-terms satisfying a condition called the global trace condition, 
and they denote total functionals. $\Reg$ is the subset of terms which are regular trees (having finitely
many subtrees), and therefore are possibly infinite terms which are finitely presented.
$\CTlambda$ is $\GTC \cap \Reg$, that is, it consists of all well-typed circular 
$\lambda$-terms satisfying satisfying the global trace condition and which are
regulal trees. We will prove that $\CTlambda$ is a decidable subset of $\Reg$.
$\CTlambda$ is a new circular version of G\"{o}del system $\systemT$. 
Differently from all previous circular version of $\systemT$, our system $\CTlambda$
uses binders instead of combinators. 
Our goal is providing a circular syntax more familiar to researchers working in the field of Type Theory.

We will prove the expected results for the circular syntax $\CTlambda$:
strong normalization for reductions not inside any fixed point and Church-Rosser. 
We will prove normalization in the limit if we use reductions which are ``fair'':
they can reduction \emph{inside} fixed points, but they never forget entirely 
to reduce \emph{outside} all fixed points.
Besides, we will prove that the closed terms $\CTlambda$ (those without free \emph{term} variables)
represent exactly the total computable functionals definable in G\"{o}del system $\systemT$
using $\N$ as only atomic type.

%14:57 17/04/2024

\section{ The set of infinite $\lambda$-terms}
We define the set $\LAMBDA$ of infinite terms, the subset $\WTyped$
of well-typed terms and a reduction relation for them.

\begin{definition}[Types of $\LAMBDA$]
\mbox{}
\begin{enumerate}

\item
The types of $\LAMBDA$ are: the type $\N$ of natural numbers, an infinite list 
$\alpha,\beta,\ldots$ of type variables, and with $A,B$ also  $A \rightarrow B$.
We call them simple types, \emph{types} for short. 

\item 
$\Type$ is the set of simples types.
\end{enumerate}
\end{definition}

We write $\Nat$ for the set of natural numbers: 
all $n \in \Nat$ are represented in $\LAMBDA$ by expressions of type $\N$ which we call numerals.

\begin{definition}[Terms of $\LAMBDA$]
\mbox{}
\begin{enumerate}

\item
The terms of $\LAMBDA$ 
are all possibly infinite trees we can define with $x^T$ (variable name with superscript $T \in \Type$), 
$\lambda x^T.t$ (binder, abstraction), 
$\ap(t,u)$, $0$, $\Succ (t)$ (successor of $t$), $\cond x^\N.(f,g)$ (binder, arithmetical conditional).

\item
$t$ is an immediate subterm of $\Succ(t)$, $f$, $g$ are  immediate subterm of $\cond x^\N.(f,g)$
and $t$, $u$ are  immediate subterm of $t(u)$.

\item 
The subterm relation is the reflexive and transitive closure of the immediate subterm relation.

\item
We write $\SubTerm(t)$ for the set of subterms of $t$. 

\item
We write $\Tree(t)$ for the tree of all chains
$(t_1, \ldots, t_n)$, with $t_1=t$ and $t_{i+1}$ immediate subterm of $t_i$ for all $(i+1) \le n$.
subterm 

\item
When $t = \Succ ^n(0)$ for some natural number $n \in \Nat$
we say that $t$ is a numeral. We write $\Num$ for the set of numeral.

\end{enumerate}
As usual, we abbreviate $\ap(t,u)$ with $t(u)$. 
\end{definition}

We use two different names for the operation $\ap(t,u)$: 
we call it $\ap$ when $u$ is not a variable and $\etaRule$ when $u$ is a variable. 
$\Num$ is the representation inside $\LAMBDA$ of the set $\Nat$ of natural numbers.
All numeral are finite terms of $\Lambda$. 
All finite well-typed typed $\lambda$-terms 
we can define with the rules above are finite terms of $\LAMBDA$.
The term $t = \cond x^\N.(0,t)$ is in $\LAMBDA$. 
The set $\SubTerm(t)=\{t,0\}$ of subterms  of $t$ is finite, therefore $t$ is a regular term.
However, $\Tree(t)$ is an infinite tree (it includes itself as a subtree). 
The finite branches of $\Tree(t)$ are all $(t,t,t,\ldots,0)$, the unique infinite branch of $\Tree(t)$
is $(t,t,t,\ldots)$. 

In order to define the type of a an infinite term, we first 
define the context for any term of $\LAMBDA$.


\begin{definition}[Types and Contexts of $\LAMBDA$]
\mbox{}
\begin{enumerate}

\item
A  context of $\LAMBDA$ is any finite list $\Gamma = ({x_1}^{A_1}, \ldots, x_n^{A_n})$ 
of pairwise distinct variables, each assigned to some type $A_1, \ldots, A_n \in \Type$. 
We write $\FV(\Gamma) = \{x_1, \ldots, x_n\}$.

\item
If $\Gamma = ({x_1}^{A_1}, \ldots, x_n^{A_n})$ ,
$\Gamma' = (x'_1:A'_1, \ldots, x'_n:A'_{n'})$ are context of $\LAMBDA$, then we
write $\Gamma \sqsubseteq \Gamma'$ if $\Gamma$ is a sub-list of $\Gamma'$, that is,
if there is a (unique) increasing map $f:\{1,\ldots,n\} \rightarrow \{1,\ldots,n'\}$
such that $x_{i}=x'_{f(i)}$ and $A_{i}=A'_{f(i)}$ for all $i \in \{1,\ldots,n\}$.

\item
If $\Gamma$ is a context of $\LAMBDA$, then $\Gamma\setminus\{x\}$ is the context obtained
by removing $x_i:A_i$ from $\Gamma$ if $x_i=x$. If $x \not \in \FV(\Gamma)$ then
 $\Gamma\setminus\{x\} = \Gamma$.

\end{enumerate}
\end{definition}


%12:38 17/04/2024
%15:37 17/04/2024

From the context for a term we can define a context for each subterm of the term.

\begin{definition}[Inherited Contexts of $\LAMBDA$]

Given any context $\Gamma$, any $t \in \Lambda$ and any subterm $u \in \SubTerm(t)$,
we define a unique inherited context for $u$, of the form $\Gamma',\Delta$ for some 
$\Gamma'\sqsubseteq \Gamma$ and some $\Delta$.

\begin{enumerate}

\item
$t$ has inherited context $\Gamma$.

\item
Any binder on $x$ subtracts the variable $x$ from the context of its \emph{last} argument:
if $t = \lambda x^T.u, \cond x^\N.(f,g)$ has inherited context $\Gamma',\Delta$, 
then $u$ and $g$ have context $(\Gamma', \Delta) \setminus \{x\}, x^T$, 
while $f$ has  inherited context 
$\Gamma',\Delta$.

\item
In any other case the context of a term and of the immediate subterm are the same:
ff $t=\Succ(u), f(a)$ have inherited context $\Gamma', \Delta$,
 then $u,f,a$ have  inherited context $\Gamma', \Delta$.
\end{enumerate}
We abbreviate \emph{`` inherited context from $\Gamma$''} with \emph{context}.
\end{definition}



%The scope of the binder $\lambda x^T.t$ is $t$.
%The scope of the binder $\cond x^\N.(f,g)$ is $g$ ($f$ is \emph{not} in the scope of $\cond x^\N.(f,g)$).

We define typing rules for terms of $\LAMBDA$ and the subset $\WTyped$ of well-typed terms.
The typing rules are the usual ones but for the conditional binder $\cond$, and for a fresh rule $\etaRule$
for typing the application $t(x)$ of a term  to a variable $x$ of type $\N$. 
$\etaRule$ corresponds to an $\eta$-expansion and it introduces a global variable name $x^T$
for the first argument of $t$. 
As we said, $\etaRule$ is but a particular case of $\ap$ rule. 
We prefer to use a separate name for $\etaRule$ because of the special role  this rule has in this paper.

%We have a single structural rule  $\struct_f$, which can be used for:
% weakening, variable permutation and variable renaming. 
We have a a single structural rule $\weak$ for extending a context $\Gamma$ to a context 
$\Gamma \sqsupseteq \Gamma'$. 
Variable permutation and variable renaming are conditionally derivable for the typing rules,
and therefore are not included as rules.


\begin{definition}[Typing rules of $\LAMBDA$]
Assume $\Gamma = {x_1}^{A_1}, \ldots, x_n^{A_n}$ is a context.
%$\Delta = y_1:B_1, \ldots, y_n:B_m$ are sequents of length $n$, $m$ respectively. Suppose
%$f:\{1, \ldots, n\} \rightarrow \{1, \ldots, m\}$ is any injection, compatible with types
%in $\Gamma$, $\Delta$: we assume $A_i = B_{f(i)}$ for all $1 \le i \le n$.

\begin{enumerate}
%\item
%$\struct_f$-rule.
%If $t: \Gamma \vdash T$ then $t[ y_{f(1)}/x_1, \ldots,  y_{f(n)}/x_n]:\Delta \vdash T$
\item
$\weak$-rule (Weakening).
If $\Gamma,\Delta \vdash t:T$, $\Gamma \sqsubseteq \Gamma'$
then $\Gamma' \vdash t: T$

\item
$\etaRule$.
If $x \not \in \FV(\Gamma)$ is a variable and $\Gamma \vdash f: \N \rightarrow T$
then $\Gamma, x^\N \vdash f(x) :  T$.

\item
$\var$-rule.
If $x^A \in \Gamma$ then $\Gamma \vdash x^A$.

\item
$\ap$-rule.
If $\Gamma \vdash f:A \rightarrow B$ and $\Gamma \vdash a:A$
and $a$ is not a variable then $\Gamma \vdash f(a) : B$.

\item
$\lambda$-rule.
If $\Gamma, x^A \vdash b: B$
then $ \Gamma \vdash \lambda x^A.b :A \rightarrow B$.

\item
$0$-rule.
$\Gamma \vdash 0: \N$

\item
$\Succ$-rule.
If $\Gamma \vdash t:\N$ then $\Gamma \vdash \Succ (t):\N$.

\item
$\cond$-rule.
If $\Gamma \vdash  f :T$ and  $\Gamma, x^T \vdash g : T$ 
then $\Gamma \vdash \cond x^\N.(f,g) : \N \rightarrow T$.
\end{enumerate}
We abbreviate $\emptyset \vdash  t:A$ with $\vdash t:A$.
\end{definition}

From the typing rules we define the proofs that a term is well-typed.

\begin{definition}[Well-typed term of $\LAMBDA$]
We write $\Pi: \Gamma \vdash t:A$, and we say that $\Pi$ is a proof of $\Gamma \vdash t:A$ if
$\Pi:\SubTerm(t) \rightarrow \Type$ is an assignment of one type $B=\Pi(\pi)$ to each
subterm chain $\pi \in \SubTerm(t)$, in such a way that:
\begin{enumerate}
\item 
$\Pi(\nil) = A$
\item
Assume $\pi$ is a subterm chain from $t$ to $u$, if $u_1, \ldots, u_n$ are the immediate subterms of $u$,
$\Gamma_1, \ldots, \Gamma_n, \Gamma$ are  the inherited contexts, and $B=\Pi(\pi)$, 
$B=\Pi(\pi \conc u_1)$, \ldots, $B=\Pi(\pi \conc u_n)$.
then for some $\Gamma' \sqsubseteq \Gamma$ we have
\[
\infer[(\weak)]
{ \Gamma \prove u : B }
{
 \infer[(\typingRule)]
{  \Gamma' \prove u : B  }
    {
      {   \Gamma_1 \prove u_1:B_1}
       { \ldots }
	  {   \Gamma_n \prove u_1:B_n}
    }
}
\]
for some $\typingRule \in \{0,\Succ,\var,\ap,\lambda,\cond\}$ 
(for some non-weakening rule $\typingRule$).
\end{enumerate}
We introduce the type judgment for a term.
\begin{enumerate}
\item
$\Gamma \vdash t:A$ is true if and only if $\Pi:\Gamma \vdash t:A$ for some $\Pi$.
\item
$t \in \LAMBDA$ is well-typed if $\Gamma \vdash t:A$ for some $\Gamma, A$. 
\item
$\WTyped$ is the set of well-typed $t \in \LAMBDA$.
\end{enumerate}
%A proof $\Pi:\Gamma \vdash  t :T$ is canonical if all $\weak$-rules
%\begin{enumerate}
%\item
% either follow some $\lambda$-rule for a term $\lambda x^T.b$, and they introduce $x^T$ in the context,
%\item
%or follow some $\cond$-rule for a term $\cond x^\N.(f,g)$, and and they introduce $x^\N$ in the context,
%\end{enumerate}
\end{definition}

\begin{enumerate}
\item
Some term in $\LAMBDA$ has no type, like $0(0)$. Some terms have more than one type.
For instance if $t=\cond x^\N.(t,t)(0)$ we can prove $x^\N \vdash t:A$ for all types $A \in \Type$. 
A term with two types has the leftmost branch infinite, as it is the case for $t$ above.
\item
We will consider terms, those satisfying the global trace condition, for which typing when it exists it is unique. 
We will prove that  
term with the global trace condition has the leftmost branch finite and therefore has a unique tye.
\end{enumerate}

%Any proof $\Pi':\Gamma \vdash  t :T$ can be turned into a canonical proof 
%$\Pi':\Gamma \vdash  t :T$. Canonical proofs when they exists are unique 
%and they can be computed from $\Gamma \vdash t:T$.

As we said, $\etaRule$ is a particular case of $\ap$. When we substitute $x^\N$ with some 
$\Gamma \vdash a:\N$ which is not a variable,  then we replace $\etaRule$ with $\ap$-rule.

An example: if $t = \cond x^\N. (0,t)$, then $\vdash t:\N \rightarrow \N$.  

Our goal is to provide a set of well-formed term for $\LAMBDA$ and interpret them as partial functionals.
Some terms, those satisfying the global trace condition (to be introduced later) will be total functionals.
Our first step is to provide reduction rules for $\LAMBDA$.

%18:14 27/03/2024

\begin{definition}[reduction rules for $\LAMBDA$]
\mbox{}
\begin{enumerate}

\item
$\reduces_\beta$: $(\lambda x^A.b)(a) \reduces_\beta b[a/x]$

\item 
$\reduces_\cond$: $\cond x^\N.(f,g)(0) \reduces_\cond f$ and
$\cond x^\N.(f,g)(\Succ (t)) \reduces_\cond g[t/x]$.

\item
$\reduces$ is the context and transitive closure of $\reduces_\beta$ and $\reduces_\cond$

\item
We say that $t \reduces_\safe u$, or that $t$ reduces safely to $u$,  
if we never reduce in proper subterms of any $\cond x^\N.(f,g)$. 
%We call \emph{unsafe} a reduction inside any $\cond x^\N.(f,g)$.

\item
A term is safe-normal if all its redexes (if any) are inside some $\cond x^\N.(f,g)$.
\end{enumerate}
\end{definition}

As example. If $u = \cond x^\N. (0, (\lambda x.u)(z) )$ is as above, then $u$ is safe-normal, because
all redexes in $u$ are of the form  $(\lambda x.u)(z)$ and inside a $\cond$. 
However, the tree form of $u$ has the following branch:
$$
u, \ \ \ 
(\lambda x.u)(z), \ \ \ 
\lambda x.u, \ \ \ 
u, \ \ \ \ldots
$$
This branch is cyclic, infinite,
and it includes infinitely many $\beta$-redexes.

The reason for forbiding
reductions inside $\cond x^\N.(f,g)$ is that thorugh $\cond$ we will-express fixed-point equations.
Therefore reductions for $\cond$ are in fact unfolding of the definition of a fixed point. 
Therefore reductions for $\cond$ can can easily loop, and they are ``unsafe". 
For this reason, we first consider the minimum possible of reductions
$\cond x^\N.(f,g)(0) \reduces_\cond f$ and
$\cond x^\N.(f,g)(\Succ (t)) \reduces_\cond g[t/x]$: those on maximal $\cond$-expression.
We consider no reduction inside the arguments $f$, $g$ of $\cond$.
In a second moment, 
by adding a restriction of \emph{fairness} on reduction strategies,
we will be able to recover strong normalization also for most ``unsafe" reductions.

Some examples of reduction.
\begin{enumerate}
\item
An example of reduction of a term $v(n)$ to a normal form. 
If $n$ is any numeral and $v = \cond x^\N. (0, (\lambda x.v(x)) )$, then 
$v(n): \N$ has infinitely many $\beta$-redexes inside 
$\cond$, therefore infinitely many unsafe reductions are possible. There are only finitely many safe reduction
from $v(n)$ instead: $v(n)$ $\cond$-reduces to $(\lambda x.v(x))(n-1)$, 
this latter $\beta$-reduces to $v(n-1)$, 
then we loop: $v(n-1)$ reduces to $v(n-2)$ in one $\cond$-step and one $\beta$-step and so forth.
After $n$ $\cond$-reductions and 
$n$ $\beta$-reductions we get $v(0)$. With one last $\cond$-reduction we get $0$ and we stop. 
\item
The term $v(n)$ above is also an example of a strongly normalizing term.
There is the unique reduction sequence from $v(n)$, because we cannot $\cond$-reduce $\lambda x.v(x)$
before assigning either $0$ or $\Succ(u)$ to $x$. 
Thus, all reductions sequences from $v(n)$ terminate in $2n+1$ steps to the normal form $0$.
\end{enumerate}


\section{The trace of the $\lambda$-terms}
%19:34 27/03/2024
Total $\lambda$-terms will be well-typed terms $\Gamma \vdash t:A$ 
satisfying a condition call \emph{global trace condition} for the canonical proof 
$\Pi:\Gamma \vdash t:A$.
In order to define the global trace condition, 
we define a notion of trace for possibly infinite $\lambda$-terms, describing how an input of type $\N$
is used when computing an output.
The first step toward a trace is defining a correspondence between atoms in the
proof that $t$ is well-typed. We need first the notion of \emph{list of argument
 types} and index of atomic types for a term.

First, an example.
Assume $t : \{{x_1}^{A_1},{x_2}^{A_2}\} \vdash B_3 \rightarrow \N$. Then the list of argument types of $t$
is $A_1, A_2, B_3$, $A_1,A_2$ are named arguments with names $x_1, x_2$, and $B_3$ is an unnamed
argument. Remark that for an open term $t$ we list as ``argument types'' also the
types of the free variables. We motivate our terminology:
in a sense, $t$ is an abbreviation of the closed term $t' = \lambda  
{x_1}^{A_1},{x_2}^{A_2}.t: (  A_1,A_2,B_3 \rightarrow \N )$, and the argument types of $t'$ are
in fact $A_1, A_2, B_3$. 
The index of an atomic argument of $t$ is any $j \in \{1,2,3\}$ such that $A_j=\N$
or $B_j=\N$ respectively.

\begin{definition}[List of argument types of a term]
Assume that $\vec{A} = A_1, \ldots, A_n$, $\vec{B}=B_{n+1}, \ldots, B_{n+m}$, 
$\Gamma = \{\vec{x}:\vec{A}\}$,
and $t: \Gamma \vdash \vec{B} \rightarrow \N$.

\begin{enumerate}
\item
The \emph{list of argument types} of $t$ is $\vec{C} = \vec{A},\vec{B}$, 
the list of types of free variables and arguments of the type $ \vec{B} \rightarrow \N$ of $t$. 

\item
$A_1, \ldots, A_n$ are named arguments with names $x_1, \ldots, x_n$, and 
$B_{n+1}, \ldots, B_{n+m}$ are unnamed arguments.

\item
An \emph{index of an atomic argument} 
of $t$ is any $j \in \{1, \ldots, n+m\}$ such that $C_j = \N$.

\end{enumerate}
\end{definition}

We now define the atom correspondence between arguments of type $\N$ of subterms of $t$
in a proof $\Pi: t:\Gamma \vdash A$ of  $\LAMBDA$. 
The definition of  atom correspondence for a syntax including binders is the main contribution of this paper. 

Two arguments of type $\N$ are in
correspondence if and only if they receive the same global input: local input are ignored.
In many cases two corresponding argument types have the same index, but if we insert or remove
free variables or arguments the index may change.
Before providing the general definition, we discuss these special cases through examples. 

We draw in the same color two arguments of type $\N$ which are in correspondence. 

\subsection{Some correspondence examples}
\begin{Eg}\label{eg:0}\rm
An example of  atom correspondence for some instance of the weakening rule.
\[
\infer[(\weak)]
{  {x_1}^\bfColor{red}{\N},{x_2}^\bfColor{blue}{\N}, x_3:\bfColor{oldgold}{\N}
					\prove t : \bfColor{orange}{\N} \rightarrow \N}
	  {   {x_1}^\bfColor{red}{\N},{x_2}^\bfColor{blue}{\N} 
					\prove t : \bfColor{orange}{\N} \rightarrow \N}
\]
\end{Eg}
Remark that the type $\N$ of the variable $x_3$ (colored in \bfColor{oldgold}{old gold}) , 
introduced by weakening, is in correspondence with no type in the rule $\weak$.

%20:13 15/04/2024
\begin{Eg}\label{eg:1}\rm
An example of  atom correspondence for the rule $\ap$.
We assume that $a$ is \emph{not} a variable.
\[
\infer[(\ap)]
{  {x_1}^\bfColor{red}{\N},{x_2}^\bfColor{blue}{\N}
					\prove f(a) : \bfColor{orange}{\N} \rightarrow \N}
    {
	  {   {x_1}^\bfColor{red}{\N},{x_2}^\bfColor{blue}{\N}
					\prove f : \bfColor{oldgold}{\N}, \bfColor{orange}{\N} \rightarrow \N}
      {\ \ \ \ \   x_1: \bfColor{red}{\N}, x_2: \bfColor{blue}{\N}
					\prove a : \N}
	}
\]
\end{Eg}
Remark that the first argument of $f$ (colored in \bfColor{oldgold}{old gold}) 
is in correspondence with no argument in the rule $\ap$.
The reason is that in the term $f(a)$,
the first argument of $f$ receives a value from the value $a$ which is local to the term $f(a)$.
However, the first argument of $f$ can be in correspondence with some argument higher in the proof. 
%13:21 15/04/2024

\begin{Eg}\label{eg:2}\rm
An example of  atom correspondence for the rule $\cond$.
\[
\infer[(\cond)]
{  {x_1}^\bfColor{red}{\N},{x_2}^\bfColor{blue}{\N}
					\prove \cond x^\N.(f,g) : \bfColor{oldgold}{\N} \rightarrow \N}
    {
      {   {x_1}^\bfColor{red}{\N},{x_2}^\bfColor{blue}{\N}
					\prove f : \N}
	  {   {x_1}^\bfColor{red}{\N},{x_2}^\bfColor{blue}{\N}, x^\bfColor{oldgold}{\N} 
					\prove g : \N}
    }
\]
\end{Eg}
Remark that the first argument of $f$ (colored in \bfColor{oldgold}{old gold}) 
is in correspondence the type of the free variable $x$ in the premise for $g$,
but it is in correspondence with no argument in the  in the premise for $f$.

\subsection{A formal definition of atom correspondence}
We define an injection $$\ins:\N,\N \rightarrow\N$$
by $\ins(a,x)=x+1$ if $x \ge a+1$, and $\ins(a,x)=x$ if $x\le a$.
The role of $\ins$ is inserting one space for a new index of argument type.

\begin{definition}[Atom correspondence in a proof of  $\LAMBDA$]
Assume $\vec{A} = A_1, \ldots, A_n$, $\vec{B}=B_1, \ldots, B_m$, $\Gamma = \vec{x}:\vec{A}$,
and $\Gamma \vdash t:\vec{B} \rightarrow \N$.

For each atom index $k$ in $t$, each immediate subterms $t'$ of $t$ 
each atom index $k'$ in $t'$ we define the relation: ``$k',t'$ the successor of $k,t$". We require:
\begin{enumerate}
\item
if $t$ is obtained by a rule $\weak$ from $t'$ 
then $k = \ins(a,k)$.
\item
if $t=f(a)$ and $a$ is not a variable and $t'=f$ 
then $k' = \ins(n,k)$. If $t'=a$ then $k'=k$ and $k' \le n$.
\item
if $t=\cond x(f,g)$ and $t'=f$ 
then $k' = \ins(n,k)$. If $t'ga$ then $k'=k$.
\end{enumerate}
We require $k = k'$ in all other cases, 
which are: $\Succ $, $\lambda$, $\etaRule$.
\end{definition}

If we move up in a $\etaRule$ the type of one free variable corresponds to the type of one argument.
In a $\lambda$-rule it is the other way round. The index of two corresponding argument types
is the same for both rules. 
Below we include some examples. 
We draw in the same color two arguments of type $\N$ which are in correspondence. 


\begin{Eg}\label{eg:3}\rm
Atom correspondence for $\etaRule$.
We assume that $x$ is a variable.
\[
\infer[(\etaRule)]
{  {x_1}^\bfColor{red}{\N},{x_2}^\bfColor{blue}{\N}, x^\bfColor{oldgold}{\N}
					\prove f(x) : \bfColor{orange}{\N} \rightarrow \N}
	  {   {x_1}^\bfColor{red}{\N},{x_2}^\bfColor{blue}{\N}
					\prove f : \bfColor{oldgold}{\N}, \bfColor{orange}{\N} \rightarrow \N}
\]
\end{Eg}
Remark that the first argument of $f$ (colored in \bfColor{oldgold}{old gold}) 
in the premise is in correspondence with the last variable type in the conclusion of the rule.

\begin{Eg}\label{eg:4}\rm
Atom correspondence for  $\lambda$-rule.
We assume that $x$ is  a variable.
\[
\infer[(\ap)]
{  {x_1}^\bfColor{red}{\N},{x_2}^\bfColor{blue}{\N}
					\prove \lambda x^\N.b : \bfColor{oldgold}{\N} \rightarrow \N}
	  {   {x_1}^\bfColor{red}{\N},{x_2}^\bfColor{blue}{\N}, x^\bfColor{oldgold}{\N}
					\prove b : \N}
\]
\end{Eg}
Remark that the first argument of $\lambda x^\N.b$ (colored in \bfColor{oldgold}{old gold}) 
in the conclusion is in correspondence with the last variable type in the premise of the rule.
\\


The correspondence on atoms in a proof $\Pi:\Gamma\vdash $ defines a graph $\Graph(\Pi)$ 
whose nodes are all pairs $(k,u)$, with $u$ subterm of $t$ and $k$ index of some atomic argument of  $u$.
We define a trace as a (finite or infinite) path in the graph $\Graph(\Pi)$.

\begin{definition}[Trace for well-typed terms in $\LAMBDA$]
\begin{enumerate}
\item
A path $\pi$ in $t$ is any list $t_0, \ldots, t_n$ of subterms of $t$ 
such that $t_0 = t$ and each $t_{i+1}$ is an immediate subterm of $t_i$. 

\item
Assume $\pi =t_0, \ldots, t_n, \ldots$ is a branch of $t$, finite or infinite. 
A finite or infinite \emph{trace} $\tau$ of $\pi$ is a list 
$\tau = (k_m,t_m), \ldots, (k_n,t_n), \ldots$ such that for each $i=m,\ldots, n,\ldots$:
\begin{enumerate}
\item
$k_i$ is the index of an atomic type of $t_i$
\item
if $i+1$ is an index of $\tau$ then $k_{i+1},t_{i+1}$ is in correspondence with $k_i, t_i$ in $\Pi$.
\end{enumerate}

\end{enumerate}
\end{definition}

%20:50 15/04/2024

\section{The circular system $\CTlambda$}
We define a subset $\CTlambda$ of the set $\WTyped$ of well-typed term,
by adding the global trace condition and regularity. 
For the terms of $\CTlambda$ we will prove
strong normalization, church-rosser for terms of type $\N$, and the fact that every term of type
$\N$ is a numeral. 
As a consequence, terms $\CTlambda$ will be interpreted as total functionals. 

From the atom correspondence we define the global trace condition and terms $\CTlambda$.

We say that a tree is regular if it has finitely many subtrees. $t=\cond x^\N.(0,t)$ is an infinite regular tree,
with subtrees: $t$, $0$.

\begin{definition}[Global trace condition and terms of $\CTlambda$]
\begin{enumerate}
\item
Assume $\tau = k_m, \ldots, k_n$ is a trace of $\pi =t_0, \ldots, t_n$ and $i=m,\ldots, n$.
$\tau$ is progressing in $i$ if $t_i=\cond x^\N.(f,g)$ for some $f$, $g$,
and $k_i$ is the index of the first \emph{unnamed} argument the $\cond$-rule, 
otherwise $\tau$ is not progressing in $i$.

\item
$t$ satisfies the global trace condition if for all infinite paths $\pi$ in $t$ 
there is some infinitely progressing path $\tau$ in $\pi$.

\item
Terms of $\CTlambda$ are all well-typed terms which are regular trees (having finitely many subtrees), 
and satisfy the global trace condition.

\end{enumerate}
\end{definition}


\section{Examples of terms of $\CTlambda$}

\subsection{The sum map}
%\Daisuke{Add sum (start)}
As a first example of term of  $\CTlambda$ we can provide a term $\Sum$ computing the sum on $\N$,
which is an infinite term defined by $\Sum = \lambda x^\N.\cond z.(x,\Succ(\Sum(x)(z)))$.
This term is regular with subterms 
$$
\Sum, \  \  \ \cond z.(x,\Succ(\Sum(x)(z))), 
\  \  \ x, \  \  \ \Succ(\Sum(x)(z)),  \  \  \ \Sum(x)(z), \  \  \ \Sum(x)
$$
This term is well-typed by the following circular derivation with a back edge from the $(\dagger)$ above to the $(\dagger)$ below.
The only infinite path contains a progressing trace, a sequence of $\N$'s colored in 
\bfColor{oldgold}{old gold}. 
\[
\infer[\lambda]{
  \vdash \Sum:\N \rightarrow \bfColor{oldgold}{\N} \rightarrow \N\ 
(\bfColor{oldgold}{\dagger})
}{
  \infer[\cond]{
    x^\N \vdash \cond z.(x,\Succ(\Sum(x)(z))): \bfColor{oldgold}{\N} \rightarrow \N
  }{
    \infer[\var]{
      x^\N \vdash x^\N
    }{}
    &
    \infer[\Succ]{
      x^\N,z^\bfColor{oldgold}{\N} \vdash \Succ(\Sum(x)(z)): \N
    }{
      \infer[\eta]{
        x^\N,z^\bfColor{oldgold}{\N} \vdash \Sum(x)(z): \N
      }{
        \infer[\eta]{
          x^\N \vdash \Sum(x): \bfColor{oldgold}{\N} \rightarrow \N
        }{
          \vdash \Sum: \N \rightarrow \bfColor{oldgold}{\N} \rightarrow \N\ 
(\bfColor{oldgold}{\dagger})
        }
      }
    }
  }
}
\]

%\Daisuke{Add sum (end)}\\

%10:43 16/04/2024

\subsection{The Iterator}
A second example. We define a term $\Iter$ of  $\CTlambda$ computing the iteration of maps on $\N$.
We define a normal term $\Iter:(\N \rightarrow \N), \N,\N \rightarrow \N$ such that
$\Iter(f,a,n)=f^n(a)$ for all $n \in \N$. 
We have to to solve the equations $\Iter(f,a,0) = a$ and $\Iter(f,a,\Succ (t)) = f(\Iter(f,a,t))$.
We solve them with $\Iter = \lambda f, a, x.I$
with $I = (\cond x^\N. (a, f(I) ))(x)$.

The term is well-typed and regular by definition. We check the global trace condition. 
We mark the last unnamed argument $\N$ of $\Iter$. The mark moves to $x$ in $I$.
 Through a $\etaRule$-rule the mark moves to the unique unnamed argument $\N$ of  
$ \cond x^\N. (a, f(I) )$.
The mark either moves to the name $x$ in the context of $a$ and there it stops, or 
it progresses and moves to $x$ in the context of $f(I)$, then in the context of $I:\N$.
%??????????????? 10:30 24/03/24
Now we loop: if the path continues forever, then after infinitely many step the mark from which we started 
progresses infinitely many times, each time it crosses a $\cond$-rule.



\subsection{The Interval Map}
A third example. We simulate lists with two variables $\nil:\alpha$ and 
$\cons:\N,\alpha \rightarrow \alpha$. We recursively define a notation for lists by $[]=\nil$,
$a @ l=\cons(a,l)$ and $[a,\vec{a}] = a @ [\vec{a}]$. We add no elimination rules for lists, though,
only the variables $\nil$ and $\cons$.

We will define a term $\Interval$ with one argument $f:\N \rightarrow \N)$ and three argument
$a^\N,x^\N,y^\N$, such that 
$$
\Interval(f,a,n,m) = [f^n(a), f^{n+1}(a), \ldots, f^{n+m}(a)]
$$ 
for all $n,m \in \Num$. 
We have to to solve the recursive equations 
$$
\Interval(f,a,x,0) = [\Iter(a,f,x)]
\ \ \ \ \ \ 
\Interval(f,a,x,\Succ (t)) = \Iter(a,f,x) @ \Interval(f,a,\Succ(x),t)
$$ 
We solve them with $\Interval = \lambda f,a.v$,
where $v:\N,\N\rightarrow \N$, 
$v = \lambda x.\cond y. (\ [w],  \  w @ v(\Succ (x),y) \ )$ and $w=\Iter(f,a,x)$.

The term is well-typed and regular by definition. We check the global trace condition.
We mark the last unnamed argument $\N$ of $\Interval$.
The mark moves to the last unnamed argument $\N$ of  
$v:\N,\N \rightarrow \N$. 
We unfold $v$ to $\lambda x.\cond y. 
(\ [w],  \  [w] @ v(\Succ (x),y)) \ )$
The mark progresses and moves to the name $y$ in the context of the body of the $\lambda$-abstraction,
then to $y$ in the context of $[w]$ or of $ [w] @ v(\Succ (x),y))$,
then in the context of $\nil$ and stops, or in the context of $w=\Iter(f,a,x):\N$, 
for which we already checked the global trace condition, or in the context of $v(\Succ (x),y)):\N$. 

Then the mark moves from the named argument $y$ of $v(\Succ (x),y):\N$ to the unique 
unnamed argument $\N$ of $v(\Succ (x))$, 
and eventually to the second argument of $v:\N,\N\rightarrow \N$. 
From $v$ we loop: after infinitely many step either we reached some $w=\Iter(f,a,x)$ 
and we find some infinite progressing trace inside it, or the
mark from which we started progresses infinitely many times through the $\cond$-rule inside $v$. 
In both cases we have the global trace condition.

We have infinitely many nested $\beta$-reduction $(\lambda x. \ldots)(\Succ (x))$.
We can remove all of them in a single step. Inside the $\beta$-redex number $k$ we obtain a sub-term
$w[\Succ (x)/x]\ldots[\Succ (x)/x]$ (substitution repeated $k$ times).
The result is $w[\Succ ^k(x)] = \Iter(f,a,\Succ ^k(x))$.
The nested substitution produce new $\beta$-reductions 
$\Iter(f,a,\Succ ^k(x)) = (\lambda x.I)(\Succ ^k(x))$ for all $k \in \N$.
This is a non-regular term: we have infinitely many pairwise different 
sub-terms $x,\Succ (x),\Succ^2(x), \Succ^3(x), \ldots$.
We need infinitely many steps to normalize all $\Iter(f,a,\Succ^k(x))$ to $f^k(I)$, 
even if we allow to reduce all $\beta,\cond$-redexes at the same time.
Also the normal form is not regular: it contains all terms $f^k(I)$ for $k \in \N$, hence
infinitely many pairwise different terms. These infinite sub-terms are of a particulary simple form, though. 
They are obtained by the repeating $k$ times the assignment $z:=f(z)$, then applying $z:=I$ once
to the result.

Apparently, $\Interval$  
is some term of $\CTlambda$ which cannot be normalized in finite time, not even if we allow
infinite parallel reductions without any "safety" restriction. 
The normal form is produced \emph{only in the limit}
and it is \emph{not regular}. If we allow to reduce infinitely many nested $\beta$-redexes in one step, also
the intermediate steps of the infinite reduction of $\Interval$ are not regular.


\section{Subject reduction for terms of $\CTlambda$ of type $\N$}
%\Daisuke{SR start}
\ldots\ldots\ldots\ldots\ldots\ldots\ldots\ldots\ldots\ldots
\\
\emph{(Here we should fill this part)}
\ldots\ldots\ldots\ldots\ldots\ldots\ldots\ldots\ldots\ldots
%\Daisuke{SR end}

%15:12 16/04/2024

\section{Subject reduction for terms of $\CTlambda$ of type $\N$}
%\Daisuke{SR start}
\ldots\ldots\ldots\ldots\ldots\ldots\ldots\ldots\ldots\ldots
\\
\emph{(Here we should fill this part)}
\ldots\ldots\ldots\ldots\ldots\ldots\ldots\ldots\ldots\ldots
%\Daisuke{SR end}

%15:12 16/04/2024

\section{Weak normalization for closed terms of $\CTlambda$ of type $\N$}
\label{section-weak-normalization}
%\Daisuke{WN start}
In this section we prove that every closed term of $\CTlambda$
 (that is, well-typed, regular and with the global trace condition) normalizes with finitely many "safe" steps
to some numeral $n \in \N$.
In the following, we explicitly write $t[x_1,\ldots,x_n]$,
when each free variable in $t$ is some $x_i$, 
and, under this notation, also write $t[a_1,\ldots,a_n]$ instead of $t[a_1/x_1,\ldots,a_n/x_n]$. 
We recall that we denote with $\Nat$ the set of all numerals, namely all terms of the form
$\Succ^n(0)$ for some $n\in \N$.

We define total terms $t$ of type $\N$, 
the closed terms of type $\N$ evaluating in at least one reduction path to a numeral, and 
we define values, which are numerals (hence closed term). If $t:A \rightarrow B$ is a closed term, then
total terms and values coincide, both are the terms mapping values to total terms. Values are always
closed terms.
An open term is total if all substitutions of free variables with values produce a closed total term.
This is the formal definition:

\begin{definition}[Values and total]
  We define values and closed total terms on values by induction on types. 
  \begin{enumerate}
  \item
    A closed term $t:\N$ is closed total
    if and only if $t \safeReduces^* u$ for some $u\in \Nat$.
  \item
    A closed term $t:\N$ is a value if and only if $t \in \Nat$.
  \item
    A closed term $t:A\rightarrow B$ is total
    if and only if $t(a)$ is total for any value $a:A$.
   \item
    Closed terms and values of type  $t:A\rightarrow B$ coincide.
  \item
    A term $t[\vec{x}]:C$, whose all free variables are $\vec{x}:\vec{A}$, is
    total if and only if $t[\vec{a}]$ is total for any values $\vec{a}$ of type $\vec{A}$.
  \end{enumerate}
\end{definition}

Closed total terms are closed by safe reductions.

\begin{lemma}\label{lem:total_value}
  \begin{enumerate}
  \item\label{lem:total_value1}
    Let $t:A$ be a closed term and $t \safeReduces u$.
    If $u$ is total, then so is $t$.
  \item\label{lem:total_value2}
    Let $t[\vec{x}]:\vec{A}\rightarrow\N$ be a term,
    whose all free variables are $\vec{x}:\vec{B}$,
    and $\vec{u}:\vec{B}$ and $\vec{a}:\vec{A}$ be closed values.
    If $t[\vec{u}]\vec{a}:\N$ is total, then $t[\vec{x}]$ is total. 
  \end{enumerate}
\end{lemma}
\begin{proof}
\begin{enumerate}

\item
  \emph{The claim \ref{lem:total_value1}} is shown by induction on $A$. 
\begin{enumerate}
\item
  We first show the \emph{base case}, namely when $A$ is $\N$.
  By the assumption and the subject reduction property, $u$ has type $A$ and is closed total.
  Then we have $t \safeReduces u \safeReduces^* n$ for some $n\in\N$. 
  Hence $t:\N$ is total.
\item
  We show the \emph{induction case}, namely when $A$ is $A_1\rightarrow A_2$.
  Take arbitrary closed value $a:A_1$. Then we have $t(a) \safeReduces u(a)$ and 
  $u(a):A_2$ is total by the assumption and the subject reduction property.
  Hence by the induction hypothesis $t(a)$ is total. 
  We obtain that $t:A_1\rightarrow A_2$ is total. 
\end{enumerate}

\item  
  The claim \ref{lem:total_value2} is shown by induction on $|\vec{A}|$.
\begin{enumerate}
\item
  The \emph{base case} $|\vec{A}| = 0$ is immediately shown by the definition.
\item
  We show the \emph{induction case}. Let $\vec{A} = A_0\vec{A'}$.
  Take arbitrary values $\vec{u}:\vec{B}$, $\vec{a'}:\vec{A'}$, and $a_0:A_0$. 
  By the assumption, we have $t[\vec{u}]a_0\vec{a'}:\N$ is closed total for all values $\vec{a'}$. 
  Then $t[\vec{u}]a_0:\vec{A'}\rightarrow\N$ is total 
  by the induction hypothesis on $\vec{A'}\rightarrow \N$.
  By definition $t[\vec{u}]:\vec{A}\rightarrow\N$ is closed total,
  and so by definition $t[\vec{x}]$ is total, as we expected.
\end{enumerate}

\end{enumerate}
\end{proof}

Let $\Pi$ be a proof and $e$ be a node of $\Pi$.
We write $\Pi(e)$ for $t:\Gamma\vdash A$ at the node $e$ in $\Pi$. 

\begin{theorem}
  Assume $\Pi:\Gamma\vdash t:A$.
  If $t$ is not total, then there is some infinite path in $\Pi$ with no infinite progressing trace. 
\end{theorem}

%19:01 16/04/2024

\begin{proof}
  Assume that $t$ is not total
  and $t: \vec{x}:\vec{D}\vdash\vec{A}\rightarrow\N$ has a proof $\Pi$ (we show a contradiction).
  By \ref{lem:total_value2} of Lemma~\ref{lem:total_value},
  there exist closed values $\vec{a}:\vec{A}$ and $\vec{d}:\vec{D}$ such that
  $t[\vec{d}]\vec{a}$ is not total. 
  We inductively construct $(e_i,\vec{d_i},\vec{a_i})$ for each natural number $i$,
  such that
  \begin{itemize}
  \item[(i)]
    $e_i$ is a node of $\Pi$, where $t_i: \vec{x_i}:\vec{D_i}\vdash\vec{A_i}\rightarrow\N$ is
    at the node $e_i$ in $\Pi$, and $e_{i+1}$ is a child node of $e_i$ in $\Pi$; 
  \item[(ii)]
    $t_i$ is not total;
  \item[(iii)]
    $\vec{d_i}:\vec{D_i}$ and $\vec{a_i}:\vec{A_i}$ are closed values
    such that $t_i[\vec{d_i}]\vec{a_i}$ is not total;
  \item[(iv)]
    there is a trace compatible connection from
    ($\vec{d_i};\vec{a_i}$) to ($\vec{d_{i+1}};\vec{a_{i+1}}$), for any $i\ge 0$, in $\Pi$.
    Moreover, if $i$ is a progress point, namely $t_i$ is of the form $\cond x^\N.(f,g)$ and $t_{i+1}$ is $g$,
    then $\vec{a_i} = \Succ(m')\vec{a'}$, $\vec{d_{i+1}} = \vec{d_i}m'$, and
    a trace passes from $\Succ(m')$ to $m'$. 
    %\Daisuke{mynote:write this more clearly}
  \end{itemize}
  
  We first define $(e_0,\vec{d_0},\vec{a_0})$ by $(r,\vec{d},\vec{a})$,
  where $r$ is the root node of $\Pi$. Then we have $t_0 = t$, (ii) and (iii). 
  Next, assume that $(e_i,\vec{d_i},\vec{a_i})$ is already constructed.
  Then we define $(e_{i+1},\vec{d_{i+1}},\vec{a_{i+1}})$ by the case analysis of
  the last rule for the node $e_{i+1}$ in $\Pi$. 

%%%%%%%%%%%%%%%%%%%%%%%%%%%%%%%%
% APPARENTLY THE CASE $\struct(f)$ CAN BE SIMPLIFIED TO WEAKENING
%%%%%%%%%%%%%%%%%%%%%%%%%%%%%%%%
  The case of $\struct(f)$, namely
  $\Pi(e_i) = t[y_{f(1)}/x_1,\ldots,y_{f(n)}/x_n]:\Delta\vdash \vec{A_i}\rightarrow\N$
  is obtained from $t:\Gamma\vdash A$, where
  $\Gamma = x_1:C_1,\ldots,x_n:C_n$, $\Delta = y_1:B_1,\ldots,y_m:B_m$, 
  $f:\{1,\ldots,n\}\to\{1,\ldots,m\}$ is an injection, and $C_i=B_{f(i)}$ for all $1\le i\le n$.
  By the induction hypothesis and $(b)$,
  $t[y_{f(1)}/x_1,\ldots,y_{f(n)}/x_n][d_{i,f(1)}/y_{f(1)},\ldots,d_{i,f(n)}/y_{f(n)}]\vec{a_i}:\N$ is not total, where $\vec{d_i} = d_{i,1}\ldots d_{i,m}$.
  Then define $e_{i+1}$ as the unique parent node of $e_i$, and
  also define $\vec{d_{i+1}}$ and $\vec{a_{i+1}}$ by $d_{i,f(1)}\ldots d_{i,f(n)}$
  and $\vec{a_i}$, respectively. 
  We obtain (i), (ii), and (iii) for $i+1$, as expected.
  We also have (iv) since the connection, determined by $f$, from
  $(d_{i,1}\ldots d_{i,m};\vec{a_i})$ to $(\vec{d_{i+1}};\vec{a_{i+1}}) = (d_{i,f(1)},\ldots d_{i,f(n)};\vec{a_i})$
  is trace compatible. 
%%%%%%%%%%%%%%%%%%%%%%%%%%%%%%%%
  The case of $\eta$-rule, namely 
  $\Pi(e_i) = f[\vec{x}](x): \Gamma,x^\N\vdash \vec{A}\rightarrow\N$ is obtained from
  $f[\vec{x}]:\Gamma\vdash N\rightarrow \vec{A} \rightarrow \N$, where
  $x\not\in\Gamma$.
  By the induction hypothesis, $f[\vec{d'}]d\vec{a_i}:\N$ is not total,
  where $\vec{d_i} = \vec{d'}d$.
  Define $e_{i+1}$ as the unique parent node of $e_i$, and
  also define $\vec{d_{i+1}}$ by $\vec{d'}$ and define $\vec{a_{i+1}}$ by $d\vec{a_i}$. 
  We obtain (i), (ii), and (iii) for $i+1$, as expected.
  We also have (iv) since the connection from
  $(\vec{d_i};\vec{a_i}) = (\vec{d'}d;\vec{a_i})$ to
  $(d_{i+1};\vec{a_{i+1}}) = (\vec{d'};d\vec{a_i})$
  is trace compatible. 

  The case of $0$-rule, namely $\Pi(e_i) = 0:\Gamma\vdash \N$, is not the case since $t_i = 0$ is total.
  
  The case of $\Succ$-rule, namely $\Pi(e_i) = \Succ(t_{i+1}):\Gamma\vdash \N$ is obtained from
  $t_{i+1}:\Gamma\vdash \N$.
  By the induction hypothesis, $\vec{a_i}$ is empty, and 
  $\Succ(t_{i+1}[\vec{d_i}]):\N$ is not total.
  Then, by the definition, $t_{i+1}[\vec{d_i}]$ is not total. 
  Define $e_{i+1}$ as the unique parent node of $e_i$, and
  also define $\vec{d_{i+1}}$ and $\vec{a_{i+1}}$ by $d_i$ and empty, respectively. 
  We obtain (i), (ii), (iii), and (iv) for $i+1$, as expected.

  The case of $\ap$-rule, namely 
  $\Pi(e_i) = t[\vec{x}](u[\vec{x}]):\Gamma\vdash \vec{A}\rightarrow\N$ is obtained from
  $t[\vec{x}]:\Gamma\vdash B \rightarrow \vec{A}\rightarrow\N$ 
and $u[\vec{x}]:\Gamma\vdash B$.
  By the induction hypothesis, $t[\vec{d_i}](u[\vec{d_i}])\vec{a_i}:\N$ is not total.
  We first consider the subcase that $u[\vec{d_i}]:B$ is total.
  We define a value $a':B$ by $n\in\Nat$ such that $u[\vec{d_i}] \safeReduces^* n$ if $B=\N$,
  by $u[\vec{d_i}]$ otherwise.
  Then define $e_{i+1}$ by the parent node whose term is $t[\vec{x}]$,
  and define $\vec{d_{i+1}} = \vec{d_i}$ and $\vec{a_{i+1}} = a'\vec{a_i}$. 
  Using Lemma~\ref{lem:total_value}, we obtain (i), (ii), (iii) for $i+1$, as expected. 
  We also have (iv) since the connection from 
  $(\vec{d_i};\vec{a_i})$ to $(\vec{d_{i+1}};\vec{a_{i+1}}) = (\vec{d_i};a'\vec{a_i})$ is
  trace compatible. 
  Next we consider the subcase that $u[\vec{d_i}]:B$ is not total.
  Then there is a sequence of values $\vec{a'}$ such that $u[\vec{d_i}]\vec{a'}:\N$ is not total.
  Define $e_{i+1}$ by the parent node whose term is $u[\vec{x}]$,
  and define $\vec{d_{i+1}} = \vec{d_i}$ and $\vec{a_{i+1}} = \vec{a'}$. 
  We obtain (i), (ii), (iii) for $i+1$, as expected.
  We also have (iv) since the connection from 
  $(\vec{d_i};\vec{a_i})$ to $(\vec{d_{i+1}};\vec{a_{i+1}}) = (\vec{d_i};\vec{a'})$ is
  trace compatible. 
  
  The case of $\lambda$-rule, namely
  $\Pi(e_i) = \lambda x.(t_{i+1}[\vec{x},x]):\Gamma\vdash A \rightarrow \vec{A} \rightarrow \N$ is obtained from
  $t_{i+1}[\vec{x},x]:\Gamma,x^A\vdash \vec{A}\rightarrow\N$.
  By the induction hypothesis, $(\lambda x.(t_{i+1}[\vec{d_i},x]))a\vec{a'}:\N$ is not total,
  where $\vec{a_i} = a\vec{a'}$.
  Then, by Lemma~\ref{lem:total_value}, $t_{i+1}[\vec{d_i},a]\vec{a'}$ is not total. 
  Define $e_{i+1}$ as the unique parent node of $e_i$, and
  also define $\vec{d_{i+1}} = \vec{d_i}a$ and $\vec{a_{i+1}} = \vec{a'}$. 
  We obtain (i), (ii), (iii) for $i+1$, as expected.
  We also have (iv) since the connection from 
  $(\vec{d_i};\vec{a_i}) = (\vec{d_i};a\vec{a'})$ to $(\vec{d_{i+1}};\vec{a_{i+1}}) = (\vec{d_i}a;\vec{a'})$ is
  trace compatible. 
  
  The case of $\cond$-rule, namely
  $\Pi(e_i) = \cond x^\N.(f[\vec{x}],g[\vec{x},x]):\Gamma\vdash \N\rightarrow\vec{A}\rightarrow\N$
  is obtained from 
  $f[\vec{x}]:\Gamma\vdash \vec{A}\rightarrow\N$
  and
  $g[\vec{x},x]:\Gamma,x^\N\vdash \vec{A}\rightarrow\N$. 
  By the induction hypothesis, $\cond x^\N.(f[\vec{d_i}],g[\vec{d_i},x])m\vec{a'}:\N$ is not total,
  where $\vec{a_i} = m\vec{a'}$ and $m \in \Nat$.
  We first consider the subcase $m=0$.
  Define $e_{i+1}$ by the parent node whose term is $f[\vec{x}]$,
  and define $\vec{d_{i+1}} = \vec{d_i}$ and $\vec{a_{i+1}} = \vec{a'}$. 
  We obtain (i), (ii), (iii) for $i+1$, as expected. 
  We also have (iv) since the connection from 
  $(\vec{d_i};\vec{a_i})=(\vec{d_i};0\vec{a'})$ to $(\vec{d_{i+1}};\vec{a_{i+1}}) = (\vec{d_i};\vec{a'})$
  is trace compatible. 
  Next we consider the subcase $m=\Succ(m')$. 
  Define $e_{i+1}$ by the parent node whose term is $g[\vec{x},x]$,
  and define $\vec{d_{i+1}} = \vec{d_i}m'$ and $\vec{a_{i+1}} = \vec{a'}$. 
  We obtain (i), (ii), (iii) for $i+1$, as expected.
  We also have (iv) since the connection from 
  $(\vec{d_i};\vec{a_i})=(\vec{d_i};\Succ(m')\vec{a'})$ to $(\vec{d_{i+1}};\vec{a_{i+1}}) = (\vec{d_i}m';\vec{a'})$
  is trace compatible. 

  Hence, by the above construction, we have an infinite path $\vec{e} = (e_0,e_1,\ldots)$ in $\Pi$.
  Since $\Pi$ satisfies the global trace condition, $\vec{e}$ contains a progressing trace
  $(k_{m},k_{m+1},\ldots)$, where, for each $m\le i$, $k_i$ is an atom index of $t_i$, 
  and $k_{i+1},t_{i+1}$ is the successor of $k_i,t_i$. 
  Let $n_i$ be an numeral in $\vec{d_i},\vec{a_i}$ at index $k_i$ for each $m\le i$.
  Then the sequence $(n_m,n_{m+1},\ldots$ decreases at each progressing point.
  This means that it decreases infinitely many times
  since $(k_{m},k_{m+1},\ldots)$ has infinitely many progressing point.
  Finally we have a contradiction. 
  
\end{proof}

By this theorem, we have the following corollaries. 

\begin{corollary}
  $t: \Gamma\vdash A$ implies $t$ is total.
\end{corollary}

\begin{corollary}
  For any closed $t:\N$, there is $n\in\Nat$ such that $t\safeReduces^* n$. 
\end{corollary}



%\Daisuke{WN end}

\section{Uniqueness of normal form for closed terms of $\CTlambda$ of type $\N$}
In this section we prove a weak form of confluence: normal form for
all closed terms of $\CTlambda$ of type $\N$ is unique.

By the weak normalization result proved in \S  \ref{section-weak-normalization},
we will deduce: for all for all closed terms $t$ of $\CTlambda$ of type $\N$, 
there is some $n \in \N$ such that there is a safe reduction $t \reduces n$, and all normal form of $t$
are equal to $n$.

We define a notion $\sim$ of equivalence for well-typed terms with the same type.
Assume $t, u : A$ are \emph{closed} terms of $\CTlambda$.
 (that is, well-typed, regular and with the global trace condition). 

We define an equivalence $\sim_0$ for closed terms,
by induction on the type $A$ of $t$, $u$.

\begin{enumerate}
\item
Assume $A = \N$. Then $t \sim_0 u$ if and only if for all normal $t',u':\N$, if $t \reduces t'$
and $u \reduces u'$ then $t'=u'$.
\item
Assume $A = B \rightarrow C$. Then $t \sim_0 u$ if and only if for all closed $b,c:B$ if
$b \sim_0 c$ then $t(b) \sim_0 u(c)$.
\end{enumerate}

We will prove that for all closed terms $t$ of $\CTlambda$ we have $t \sim_0 t$.
If $A=\N$ this means that the normal form of all closed terms of $\CTlambda$ of type $\N$ is unique,
which is our goal.

We first define $t \sim u$ for \emph{any} terms of $\CTlambda$ with the same type.
 $t \sim u$ if and only if for any two substitution 
$\sigma = [\vec{t} / \vec{x}]$ and $\tau = [\vec{u} / \vec{x}]$,
if $\vec{t} \sim_0 \vec{u}$ and $\sigma(t)$, $\sigma(u)$ are closed then $\sigma(t) \sim_0 \sigma(u)$.

We will prove that  for all terms $t$ of $\CTlambda$ we have $t \sim t$. In the case of closed terms $t$
we can choose $\sigma=\tau=$ the empty sbustitution and we obtain $t \sim_0 t$, which is our goal.

Assume that $t:\N$ is a closed term $t$ of $\CTlambda$ and $n \in \N$. 
\begin{enumerate}
\item
We say that $n$ is a value of $t$ if $t \reduces n$ (if $t$ safely reduces to $n$). 
\item
We say that $t$ is confluent if $t \sim_0 t$.
\end{enumerate}
Assume $\vec{x}:\vec{A}$ and $t[\vec{x}]: \vec{D} \rightarrow \N$. Assume 
$\vec{a}:\vec{A},\vec{d}:\vec{D}$ 
and $\vec{b}:\vec{A},\vec{e}:\vec{D}$ are vectors of closed terms of $\CTlambda$. We say that 
these four vectors are a counterexample to $t$ if $\vec{a},\vec{d} \sim_0 \vec{b},\vec{e}$
and $\neg (t[\vec{a}](\vec{d}) \sim_0  t[\vec{b}](\vec{e}))$.

We have $t \sim t$ if and only if there is no counter-example for $t$.


By the weak normalization result proved in \\Succ  \ref{section-weak-normalization}, all 
closed term $t$ of $\CTlambda$ of type $\N$ have a value. If $t$ is confluent, then the value is unique.
For confluent closed term $t$ of $\CTlambda$ of type $\N$, "safe" reduction preserves the value. Indeed,
if $t \reduces u$ then $u$ is a closed term of $\CTlambda$ of type $\N$, therefore $u$ has a value
$m$, which is also a value of $t$. By confluence of $t$ we conclude that $n=m$.

If $t$ is as above, with value $n$, and $t \reduces \Succ (u)$, then $n>0$, 
$u$ is a confluent closed term $t$ of $\CTlambda$ of type $\N$, and the value of $u$ is $n-1$.
Indeed,if $u \reduces u'$ and $u'$ is normal, then $t \reduces \Succ (u')$, therefore $\Succ (u')=n$ 
by confluence of $t$, and we conclude that $n>0$ and $u'=n-1$. Thus, the normal form of $u$ is unique.

We will prove the following.

\begin{theorem}
Assume $t$ is any well-typed term such that $\neg (t \sim t)$. Then there is some infinite path 
$\pi=\{t_i | i \in \N\}$ of $t$ with some infinite sequence $\sigma= \{\sigma_i| i \in \N\}$, with
each $\sigma_i$ counter-example for $t_i$, and with the value of $\sigma$ compatible with the
trace conditions of $\pi$.
\end{theorem}

As a corollary from the theorem we will conclude: 
if $t$ satisfies the global trace condition, then there is no such $\sigma$, therefore $t \sim t$, 
as we wished to show.

We prove that for any term $ (t \sim t)$ with counter-example $\vec{a},\vec{d} \sim_0 \vec{b},\vec{e}$
and $\neg (t[\vec{a}](\vec{d}) \sim_0  t[\vec{b}](\vec{e}))$ we can find some immediate
subterm $t'$ and with a counter-example $\vec{a'},\vec{d'} \sim_0 \vec{b'},\vec{e'}$
and $\neg (t'[\vec{a'}](\vec{d'}) \sim_0  t'[\vec{b'}](\vec{e'}))$, \emph{compatible with trace
condition}.


%21:09 20/03/2024

%
%\[
%	\infer[(\Case\ E)]{Ex \prove \N x}{
%		\infer[(=L)]{x=0 \prove \N x}{
%		\infer[(Wk)]{x=0 \prove \N 0}{
%		\infer[(\N\ R)]{\prove \N 0}{}
%		}}
%		&
%		\infer[(=L)]{x=sx',Ox' \prove \N x}{
%		\infer[(\N\ R)]{x=sx',Ox' \prove \N sx'}{
%		\infer[(Wk)]{x=sx',Ox' \prove \N x'}{
%		\infer[(\Subst)]{Ox' \prove \N x'}{
%		Ox \prove \N x
%		}}}}
%	}
%\]
%

\section{Infinite reductions}
For infinite terms we can define a notion of reduction contracting infinitely many redexes at the same time.
In this section investigate this notion, 
we show that for recursive terms it is a recursive operation, and that the result
is well-defined for the set $\GlobalTraceCondition$ of terms with the global trace condition. 
We also show that $\GlobalTraceCondition$ is closed under infinite reductions.

Assume $t \in \LAMBDA$ and $X$ is a set of redexes in $t$. We define an operation $\reduces_X$ 
on each node $t'$ of $t$, and its result $u \in \LAMBDA$ by induction on the distance between
$t'$ and $t$.

If $t \not \in X$ and $t=0,\Succ(u), x, \lambda x^Tu, f(u), \cond(f,g)$, we apply $\reduces_X$
on all immediate subterms $t'$ of $t$. If $t = (\lambda x^T.b)(u), \cond x^\N.(f,g)(0), \cond x^\N.(f,g)(\Succ(u))$,
we contract $t$, respectively, to $b[u/x], f, g[u/x]$, then we apply $\reduces_X$ to the result.

The operation $\reduces_X$ loops if in some path of $t$ we always find redexes in $X$. This is the case of 
$t = (\lambda x^T.t)(x)$ and of $u=\cond x^\N.(0,u)(1)$ and of $X=$ all redexes of $t$, $u$ respectively. 
In both terms, the unique infinite path of the term is made of redexes of $X$, and $\reduces_X$ never
provides a single symbol of the output.

If we want to consider infinite redexes in $\LAMBDA$ we should allow undefined sub-terms inside a term.
This is not the case for the set $\GlobalTraceCondition$ of terms with the global trace condition:
we can show that $\GlobalTraceCondition$ is closed under $\reduces_X$.

We first check that $\reduces_X$ applied to $\GlobalTraceCondition$ produces no undefined sub-term.

\begin{lemma}[No infinite rexed path]
If $t \in \GlobalTraceCondition$ (if $t$ has the global trace condition)
then $\reduces_X$ applied to $t$ produces no undefined sub-term.
\end{lemma}

\begin{proof}
We have to prove that there is no path in $t$ only made  of redexes.
Suppose a path in $t$ is only made  of redexes 
$r=(\lambda x^T.b)(u)$ or $r=(\cond x^\N.(f,g))(t)$, where $t=0$ or $t=\Succ(u)$. 
We first check that no trace $\tau$ starting from $r$ can progress in the next sub-term of the path.
Indeed, if $\tau$ progresses, then $r=(\cond x^\N.(f,g))(t)$
and $\tau$ to the type of the variable $x$ bound by $\cond x^\N.(f,g)$. 
The only way to obtain this is that $t=x$ and $r$ is obtained by an $\etaRule$,
and in this case $r$ is no redex, because $x \not = 0, \Succ(u)$.

This implies that in a path $\pi$ only made of redexes any trace $\tau$ 
cannot progress after its first point, and therefore it cannot infinitely progress and the term has
no global trace condition.
\end{proof}


We prove that $\GlobalTraceCondition$ is closed under $\reduces_X$.


\begin{lemma}
If $t \in \GlobalTraceCondition$ and $t \reduces_X u$ then $u \in \in \GlobalTraceCondition$
\end{lemma}

\begin{proof}
We prove that for any infinite path $\pi$ in $u$ there is some infinite path $\rho$ in $t$ such that
for all traces $\tau$ of $\rho$ there is some trace $\sigma$ of $\pi$ which is obtained by
removing some points of $\tau$ which are either the first point or are not progress points. 
It will follow that if we have an infinitely progressing trace $\tau$ in $\rho$, then we have some infinitely
progressing trace $\sigma$ in $\pi$.

We follow $\pi$ and we apply $\reduces_X$. We find finitely many redexes which we reduce to
some $v[t_n/x_n]\ldots[t_1/x_1]$ with $v$ not redex. A single step in $\rho$ through $v$
corresponds to $n+1$ steps in $\pi$, in such a way that a trace in $\pi$ is restricted to a trace
in $\rho$. The restriction removes no progresses point but at most the first. 

If the path $\rho$ continues in this way forever then we are done.

The other possibility is: $\pi$ could move to some $t_i$ which was an $x_i$ in $t$. In this case we restart
the path in $t$ through $t_i$, while $\rho$ continues in $t_i[t_{i-1}/x_{i-1}]\ldots[t_1/x_1]$.
After finitely many steps $\rho$ can continue through $t_j$ for some $1 \le j < i$, which was an $x_j$
in $t$. This process can continue at most $n$ times. 
Eventually $\rho$ continues forever inside $t_k$ for some $1 \le k \le n$. In this case there is a $1$ to $1$
correspondence between the trace in $\rho$ and in $t_k$ and the traces in $\pi$ and in $t_k$.
\end{proof}


\section{Normalization and Fairness}


\ldots\ldots\ldots\ldots\ldots\ldots\ldots\ldots\ldots\ldots
\\
\emph{(Here we should insert our notes for the proof of strong normalization for safe reductions)}
\ldots\ldots\ldots\ldots\ldots\ldots\ldots\ldots\ldots\ldots
\\

There are terms requiring infinite reductions to normalize, but there are reductions normalizing in the limit.
We conjecture that we can characterize reductions normalizing in the limit using the notion of "fair" reductions.

\begin{definition}
\begin{enumerate}
\item
We say that a node is in the $k,\cond$-level if it has less than $k$ nodes $\cond$ in its branch.
\item
We say that an infinite reduction sequence is is \emph{fair} for nodes of $k,\cond$-level in the following case:
for all $i\in \N$, all $k>0$, there is some  $j \ge i$ such that at step $j$ either all nodes in the 
$k,\cond$-level are normal, or one of them is reduced at step $j$.
\end{enumerate}
\end{definition}

\emph{Conjecture 1}. For all $k \in \N$, an infinite reduction sequence $\sigma$ 
eventually stop reducing nodes in the $k,\cond$-level.

Conjecture 1 should be proved by adapting the proof of strong normalization for safe reductions to a
termination result on the first $k$-levels for all reductions, and assuming the same result for $k-1$.

\emph{Conjecture 2}. An infinite reduction sequence $\sigma$ 
is normalizing in the limit if and only if for all $k \in \N$,
$k > 0$, $\sigma$ is fair for the task of reducing nodes in the $k,\cond$-level.

Conjecture 2 should follow in few steps from conjecture Conjecture 1.



\section{Equivalence between cyclic and non-cyclic system $T$} 

In this section we prove that the set of total functionals on $\N$ definable in 
$\systemT$ and $\CTlambda$ are the same.

The inclusion from $\systemT$ and $\CTlambda$ is easy to prove.
For any type $T$ we can define a term $\Rec:T,(\N,T \rightarrow T)\rightarrow T$ such that
$\Rec(a,f,0) = a$ and $\Rec(a,fn+1) = f(n,\Rec(a,f,n))$, for all numeral $n \in \Nat$.
The definition is $\Rec = \lambda a,f.\rec$. with $\rec = cond x.(a,f(x,\rec(x))) : \N \rightarrow T$.

The opposite inclusion, from $\CTlambda$ to $\systemT$, it has been proved by proof-theory for the
combinatorial version of circular $\systemT$. For $\CTlambda$, we will define instead an algorithm
taking an infinite term in  $\CTlambda$, described as a finite circular tree, 
and returning a term in $\systemT$. Our translation can expand
more than exponentially the size of the finite circular tree.

Assume $t:T$ is a cyclic term, represented as a cyclic tree with buds $t_1$, \ldots, $t_n$.
We translate it to a term of $\systemT$. This is a first draft about how to do it.
 
\begin{enumerate}

\item
We first move all buds to the same type and context, by adding dummy variables and dummy arguments.
This operation preserves regularity and global trace condition.
Now $t_1$, \ldots, $t_n$ all have context $\Gamma$ and type $A$.

\item
We merge all buds into the same term, defined by some $u$ such that $u(i)=t_i$, for $i=1, \ldots, n$,
and $u(i)=$ some dummy term of type $A$ otherwise. We replace each $t_i$ with $u(i)$, 
for $i=1, \ldots, n$.
This operation preserves regularity and global trace condition.
Now we have $n$ buds, all are the same $u$ with context $\Gamma$ and type $\N \rightarrow A$.
Each bud $b$ defines a partial bijection between the occurrence of $\N$ in its context and type
$\Gamma \vdash \N \rightarrow A$, and the occurrences of $\N$ in the context and type
$\Gamma \vdash \N \rightarrow A$ of its companion. 
We extend this partial bijection to any total bijection $\tau$, depending on the but $b$.

\item
We close the partial bijections defined by each bud by composition. The number of partial 
bijections can grow in an exponential  way.

\item
Assume we have $m$ occurrences of $\N$ inside the context and type 
$\Gamma \vdash \N \rightarrow A$ of $u$.
We fix a permutation $\sigma:\{1, \ldots, m\}$ 
and we label them by variables $x_1, \ldots, x_n$ of $\systemT$,
with $x_i$ label of the argument with type $\N$ and number $i$.
We will define a translation $t^\sigma \in \systemT$ of  $t \in \CTlambda$.

\item
All traces move from $u$ to any of the occurrences of $u$ inside $u$. 
Some traces of some $\N$
in $\Gamma \vdash \N \rightarrow A$ disappear, some other are moved to some other $\N$,
in an injective way. Two traces never merge.
We label each trace in the bud $u$ with the name $x_i$ of the corresponding trace, if any.
All those corresponding to no trace are labeled at random using the remaining variable names.

At least one trace progresses, otherwise by repeating infinitely many times this step we would get a
path with no progressing trace. The same is true for any combination of one or more movements
from $u$ to $u$. 

%After $m$ movements to any $u$ inside $u$, 
%each of the $m$ traces either disappeared or cycles. After $m!$ steps, all
%cycles are back to their original point. 
%
%All traces are now restarted or move from one $\N$ to the same $\N$, with or without progression.

\item
At least one trace $x_i$ progresses and it is not erased by any other trace. Otherwise we could follow a path
in which each progress is erased in some new step, and so there is no infinite progressing trace.
We use this trace as the main variable $x_i$ of the recursion. In all steps, either $x_i$ is constant or decreases,
and in at least one case it decreases. In all cases in which $x_i$ decrease we use primitive
recursion on $x_i$ in $\systemT$, as main variable. 
In all other case, $x_i$ is not removed, therefore it stays the same. 
We isolate the main variable $x_j$ of the recursion for these steps, it is progressing therefore $j \not = i$.
We use primitive recursion on $x_j$: this is the second variable of primitive recursion. 
We continue in this way and we define a primitive recursion in $\systemT$, with pairwise distinct 
indexes $x_{i_1} = x_i$, $x_{i_2} = x_j$, \ldots, $x_{i_k}$ for some $k \ge 1$. We extend 
$x_{i_1}, \ldots, x_{i_k}$ to $x_{i_1}, \ldots, x_{i_n}$ in a random way: we defined in this way a
permutation $\sigma$ on $\{1, \ldots, m\}$ by $\sigma(j) = i_j$ for $j \in \{1, \ldots, m\}$
We define in this way a closed primitive recursive term 
$\lambda \vec{x}.t^\sigma \in \systemT$. Each bud $u$
defining a permutation $\tau$ is replaced by $\exch_{\tau}(f)$.
The term $\exch_\tau \in \systemT$ applies the permutation $\tau$ to the arguments of $f$,
and during the recursive call $f$ is replaced by $\lambda \vec{x}.u^\sigma$.
\end{enumerate}

We claim that the infinite term $t^\sigma \in \systemT$ 
is equivalent to the cyclic recursive term $t \in \CTlambda$ we started from.

%
%\section{appendix}
%
%\begin{verbatim}
%
%To: kmr@is.sci.toho-u.ac.jp (Daisuke Kimura)
%Re: proof of Weak Normalization to an integer for CT-lambda
%Fri, 22 Mar 2024 08:25:57 +0100 
%
%    By the way, I re-checked the weak curry-howard proof, now i think that the proof 
%does not require the property p-->q, a-->b ==> p[a/x]-->q[b/x] and can be completed 
%with the notion of safe reduction.
%but in fact it would be more interesting to prove full church-rosser for Circular T-lambda, 
%as anupam does for his circular T.
%
%    About strong normalization, we can prove it for "safe" reductions, those inside no cond. 
%More in general, we know that we can have infinite reduction sequences, because we can 
%have infinitely many redexes. However, for any infinite reduction sequence sigma, I conjecture 
%we can prove a kind of stabilization of the term. After some reduction step, the term only 
%changes inside some cond nested k times. 
%
%    Namely, I conjecture that
%
%"for any cyclic lambda term t, any infinite reduction sequence (sigma(n)|n in N) with sigma(0)=t, 
%any k in N, there is a n0 in N such that for all n>=n0, the terms sigma(n) and sigma(n0)  
%coincide on all branches with at most k times cond."
%
%    Best, Stefano
%
%\end{verbatim}






\end{document}
