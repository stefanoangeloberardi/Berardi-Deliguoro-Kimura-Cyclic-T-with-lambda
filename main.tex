% lmcs, 2018.5.2, 2019.9.9, 2020.2.10



\newif\ifdraft \draftfalse
\drafttrue % put % for lipics

\long\def\Stefano#1{{{\color{red}{SB: #1}}}}
\long\def\Makoto#1{{{\color{blue}{MT: #1}}}}

\ifdraft

\documentclass{article}
\usepackage{mystyle}
\A4page

\else

\documentclass{lmcs}
\usepackage{hyperref}
%\usepackage{amsmath}heoremstyle{plain}\newtheorem{satz}[thm]{Satz} %\crefname{satz}{Satz}{S\"atze}

\fi

\usepackage{mymath,proof,latexsym}
\usepackage{xcolor}

\newtheorem{theorem}{Theorem}[section]
\newtheorem{lemma}[theorem]{Lemma}
\newtheorem{proposition}[theorem]{Proposition}
\newtheorem{corollary}[theorem]{Corollary}
\newtheorem{definition}[theorem]{Definition}

\newenvironment{proof}[1][Proof]{\begin{trivlist}
\item[\hskip \labelsep {\bfseries #1}]}{\end{trivlist}}
\newenvironment{example}[1][Example]{\begin{trivlist}
\item[\hskip \labelsep {\bfseries #1}]}{\end{trivlist}}
\newenvironment{remark}[1][Remark]{\begin{trivlist}
\item[\hskip \labelsep {\bfseries #1}]}{\end{trivlist}}


\begin{document}
\sloppy 
\hbadness=10000
\vbadness=10000

% cut note 2

\def\Expand{{\rm expand}}

% cut elim

\def\RCLKIDomega{{{\bf RCLKID}^\omega}}
\def\Ren{{\rm Ren}}
\def\Terms{{\rm Terms}}
\def\IE{{\rm IE}}
\def\Cut{{\rm Cut}}
\def\Rcut{{\rm Rcut}}

% cmcs

\def\Co{{\rm co}}
\def\Axiom{{\rm Axiom}}
\def\Wk{{\rm Wk}}
\def\Cut{{\rm Cut}}
\def\BS{{\rm BS}}
\def\Bit{{\rm Bit}}
\def\LK{{\rm LK}}
\def\Head{{\rm head}}
\def\Tail{{\rm tail}}

% intuitionistic

\def\Univ{{\rm Univ}}
\def\Left{{\rm Left}}
\def\Right{{\rm Right}}
\def\Set{{\rm set}}
\def\ES{{\rm ES}}
\def\PC{{\rm PC}}
\def\Fin{{\rm Fin}}
\def\CLJIDomega{{{\bf CLJID}^\omega}}
\def\CLJIDHA{{{\bf CLJID}^\omega+{\bf HA}}}
\def\LJIDHA{{{\bf LJID}+{\bf HA}}}
\def\LJID{{\bf LJID}}
\def\OP{{\rm OP}}
\def\Div{{\rm Div}}
\def\RP{{\rm RP}}
\def\MS{{\rm MS}}
\def\ET{{\rm ET}}
\def\LiftedTree{{\rm LiftedTree}}
\def\First{{\rm first}}
\def\Er{{\rm Er}}
\def\ER{{\rm ER}}
\def\Insert{{\rm insert}}
\def\ET{{\rm ET}}
\def\SInd{{\rm SInd}}
\def\Trans{{\rm Trans}}
\def\HA{{\bf HA}}
\def\DM{{\rm DM}}
\def\Ext{{\rm ext}}
\def\KB{{\rm KB}}
\def\DT{{\rm DT}}
\def\DS{{\rm DS}}
\def\Monoseq{{\rm Monoseq}}
\def\Eq{{\rm seq}}
\def\Last{{\rm last}}

% equivalence cyclic proofs

%\def\Uor{\displaystyle \mathop {\widetilde\bigvee}}
\def\Uor{\biguplus}
\def\CLKIDomega{{{\bf CLKID}^\omega}}
\def\CLKIDPA{{{\bf CLKID}^\omega+{\bf PA}}}
\def\LKIDPA{{{\bf LKID}+{\bf PA}}}
\def\PA{{\bf PA}}

% simple equiv note

\def\LKIDN{{\bf LKIDN}}
\def\CLKIDN{{\bf CLKIDN}}

% equiv note

\def\Choose{{\rm Choose}}
\def\Asym{{\rm Asym}}
\def\Peano{{\bf Peano}}
\def\Code#1{{\lceil{#1}\rceil}}
\def\Ineq{{\rm Ineq}}
\def\MonoPath{{\rm MonoPath}}
\def\BoundedPath{{\rm BoundedPath}}
\def\InfPath{{\rm InfPath}}
\def\InfMonoPath{{\rm InfMonoPath}}
\def\HomSeq{{\rm HomSeq}}
\def\InfHomSeq{{\rm InfHomSeq}}

\def\ErdosTree{{\rm ErdosTree}}
\def\KB{{\rm KB}}
\def\MonoList{{\rm MonoList}}
%\def\Seq{{\rm Seq}}
\def\Univ{{\rm univ}}
\def\Infinite{{\rm Infinite}}
\def\S{{\bf S}}
%\def \N {{N}} %\def\N{{\bf N}} %
\def\Ind{{\rm Ind}}
\def\LKID{{\bf LKID}}
\def\PA{{\bf PA}}
\def\LKIDExt{{\bf LKIDExt}}
\def\CLKID{{\bf CLKID^\omega}}

% LICS

\def\Over#1#2{\deduce{#2}{#1}}
\def\List{{\rm List}}
\def\ListX{{\rm ListX}}
\def\ListO{{\rm ListO}}
\def\ListE{{\rm ListE}}
\def\Nl{{\rm nl}}

% CSLID

\def\SLLID{{\rm SLID}}
\def\SLID{{\rm SLID}}
\def\CSLID{{\rm CSLID}}
\def\CSLIDomega{{{\rm CSLID}\omega}}
\def\Eclass{{\rm Eclass}}
\def\Eq{{\rm Eq}}
\def\Deq{{\rm Deq}}
\def\Satom{{\rm Satom}}
\def\Cells{{\rm Cells}}
\def\Roots{{\rm Roots}}
\def\Elim{{\rm Elim}}
% \def\Case#1{{\rm Case-$#1$}}
\def\Case{{\rm Case}}
\def\Unfold{{\rm Unfold}}
\def\Pred{{\rm Pred}}
\def\Start{{\rm Start}}
\def\Unsat{{\rm Unsat}}
\def\Split{{\rm Split}}
\def\Underscore{\underline{\phantom{x}}}
\def\Rootcell{{\rm rootcell}}
\def\Rootshape{{\rm rootshape}}
\def\Jointleaf{{\rm jointleaf}}
\def\Jointleafimplicit{{\rm jointleafimplicit}}
\def\Jointnode{{\rm jointnode}}
\def\Directjoint{{\rm directjoint}}

% cyclicnote2

\def\Range{{\rm Range}}
\def\Xstart{X_{{\rm start}}}
\def\kstart{k_{{\rm start}}}
\def\Leaves{{\rm Leaves}}
\def\PureDist{{\rm PureDist}}
\def\Pure{{\rm Pure}}
\def\Identity{{\rm Identity}}
\def\Subst{{\rm Subst}}

% weak establish

\def\Connected{{\rm Connected}}
\def\Eststablished{{\rm Eststablished}}
\def\Valued{{\rm Valued}}

% monadic new

\def\SLRDbtw{{\rm SLRD}_{btw}}
\def\BTW{{{\rm BTW}}}
\def\Loc{{{\rm Loc}}}
\def\Ne{{\ne}}
\def\Nil{{{\rm nil}}}
\def\eqDef{=_{{\rm def}}}
\def\Inf#1{{\infty_{#1}}}
\def\Stores{{\rm Stores}}
\def\SVars{{{\rm SVars}}}
\def\Val{{{\rm Val}}}
\def\MSO{{{\rm MSO}}}
\def\Sep{{{\rm Sep}}}
\def\Septwo{{{\rm SLMI}}}
\def\SLMI{{{\rm SLMI}}}
\def\Sepinf{{\rm Sep}\infty}
\def\THeaps{{\rm THeaps}}
\def\Root{{\rm Root}}
\def\TG{{\rm TGraph}}

% monadic

\def\Nil{{{\rm nil}}}
\def\eqDef{=_{{\rm def}}}
\def\Inf#1{{\infty_{#1}}}
\def\Stores{{\rm Stores}}
\def\SVars{{{\rm SVars}}}
\def\Val{{{\rm Val}}}
\def\MSO{{{\rm MSO}}}
\def\Sep{{{\rm sep}}}
\def\THeaps{{\rm THeaps}}
\def\Root{{\rm Root}}
\def\TG{{\rm TGraph}}

\def\Tilde{\widetilde}
\def\Bar{\overline}
\def\Lequiv{\Longleftrightarrow}
\def\Lto{\Longrightarrow}
\def\Lfrom{\Longleftarrow}
\def\Norm{{\rm Norm}}
\def\Noshare{{\rm Noshare}}
\def\Roots{{\rm Roots}}
\def\Forest{{\rm Forest}}
\def\Var{{\rm Var}}
\def\Range{{\rm Range}}
\def\Cell{{\rm Cell}}
%\def\Tree{{\rm Tree}}
\def\Switch{{\rm Switch}}
\def\All{{\rm All}}
\def\To{\leadsto}
\def\tree{{\rm tree}}
\def\Paths{{\rm Paths}}
\def\Finite{{\rm Finite}}
\def\Const{{\rm Const}}
\def\Leaf{{\rm Leaf}}
\def\LeastElem{{\rm LeastElem}}
\def\LeastIndex{{\rm LeastIndex}}
\def\WSnS{{{\rm WSnS}}}
\def\Expand{{{\rm Expand}}}

% previous

\long\def\J#1{} % skip
\def\T#1{\hbox{\color{green}{$\clubsuit #1$}}}
\def\W#1{\hbox{\color{Orange}{$\spadesuit #1$}}}

\def\Node{{{\rm Node}}}
\def\LL{{{\rm LL}}}
\def\DSN{{{\rm DSN}}}
\def\DCL{{{\rm DCL}}}
\def\LS{{{\rm LS}}}
\def\Ls{{{\rm ls}}}

\def\FPV{{{\rm FPV}}}
\def\Lfp{{{\rm lfp}}}
\def\IsHeap{{{\rm IsHeap}}}

\def\Equiv{\quad \equiv\quad }
\def\Null{{{\rm null}}}
\def\Emp{{{\rm emp}}}
\def\If{{{\rm if\ }}}
\def\Then{{{\rm \ then\ }}}
\def\Else{{{\rm \ else\ }}}
\def\While{{{\rm while\ }}}
\def\Do{{{\rm \ do\ }}}
\def\Cons{{{\rm cons}}}
\def\Dispose{{{\rm dispose}}}
\def\Vars{{{\rm Vars}}}
\def\Locs{{{\rm Locs}}}
\def\States{{{\rm States}}}
\def\Heaps{{{\rm Heaps}}}
\def\FV{{{\rm FV}}}
\def\True{{{\rm true}}}
\def\False{{{\rm false}}}
\def\Dom{{{\rm Dom}}}
\def\Abort{{{\rm abort}}}
\def\New{{{\rm New}}}
\def\W{{{\rm W}}}
\def\Pair{{{\rm Pair}}}
\def\Lh{{{\rm Lh}}}
\def\lh{{{\rm lh}}}
\def\Elem{{{\rm Elem}}}
\def\EEval{{{\rm EEval}}}
% \def\BEval{{{\rm BEval}}} % not used 
\def\PEval{{{\rm PEval}}}
\def\HEval{{{\rm HEval}}}
\def\EVal{{{\rm Eval}}}
\def\Domain{{{\rm Domain}}}
\def\Exec{{{\rm Exec}}}
\def\Store{{{\rm Store}}}
\def\Heap{{{\rm Heap}}}
\def\Storecode{{{\rm Storecode}}}
\def\Heapcode{{{\rm Heapcode}}}
\def\Lesslh{{{\rm Lesslh}}}
\def\Addseq{{{\rm Addseq}}}
\def\Separate{{{\rm Separate}}}
\def\Result{{{\rm Result}}}
\def\Lookup{{{\rm Lookup}}}
\def\ChangeStore{{{\rm ChangeStore}}}
\def\ChangeHeap{{{\rm ChangeHeap}}}
\def\Wand{\mathbin{\hbox{\hbox{---}$*$}}}
%\def\Eval#1{[\kern -1pt[{#1}]\kern -1pt]}
\def\Eval#1{\llbracket{#1}\rrbracket}
\def\Vec{\overrightarrow}

\def\Tilde{\widetilde}
\def\Break{\hfil\break\hbox{}}

%Macro for the work on Cyclic System T with lambda
%22/03/2024
\newcommand{\ap}                { {\tt ap} }
\newcommand{\cond}            { {\tt cond} }
\newcommand{\reduces}       { \mathbin{\mapsto} }
\newcommand{\CTlambda}    { { {\tt CT} \mbox{-} \lambda} }
%\newcommand{\nameIn}       { {\tt name\mbox{-}in} }
\newcommand{\etaRule}       { {\eta\mbox{-rule}} }
\newcommand{\ins}               { {\tt ins} }
\newcommand{\struct}          { {\tt struct} }
\newcommand{\Iter}              { {\tt It} }
\newcommand{\iter}               { {\tt it} }
\newcommand{\Interval}       { {\tt In} }
\newcommand{\interval}        { {\tt in} }
\newcommand{\Succ}             { {\tt S} }
\newcommand{\nil}                { {\tt nil} }
\newcommand{\cons}             { {\tt cons} }
\newcommand{\outline}  [1]   { {\contour{black}{\textcolor{white}{#1}}}  }
\newcommand{\systemT}       { {\outline{\tt T}} }
\newcommand{\LAMBDA}       { {\outline{$\Lambda$}} }
\newcommand{\Rec}               { {\tt Rec} }
\newcommand{\rec}                { {\tt rec} }
\newcommand{\exch}             { {\tt exch} }
\newcommand{\composed}     { {\mbox{\tiny $\circ$}} }
\newcommand{\WTyped}        { {\tt WTyped} }
\newcommand{\safe}              { {\hbox{\scriptsize \tt safe}} }
\newcommand{\GlobalTraceCondition}  { {\tt GlTrCon} }
\newcommand{\Sum}              { {\tt Sum} }
\newcommand{\safeReduces} {\mathbin{\reduces_\safe}}
\newcommand{\safeReducesAst} {\mathbin{\reduces_\safe^*}}
\newcommand{\var}                { {\tt var} }
\newcommand{\Graph}            { {\tt Graph} }
\newcommand{\bfColor} [2]     {{\bf \color{#1}{#2}}}
\newcommand{\weak}             {{\tt weak}}

%colors from the web site: https://latexcolor.com/
\definecolor{amber}{rgb}{1.0, 0.75, 0.0}
\definecolor{oldgold}{rgb}{0.81, 0.71, 0.23}

\newcommand{\SubTerm}               {{\tt SubT}}
\newcommand{\Tree}                      {{\tt Tree}}
\newcommand{\N}                           {{N}}
\newcommand{\Num}                      {{\tt Num}}
\newcommand{\Nat}                        {{\outline{\tt N}}}
\newcommand{\GTC}                       {{\tt GTC}}
\newcommand{\Reg}                       {{\tt Reg}}
\newcommand{\Type}                      {{\tt Type}}
\newcommand{\conc}                      {{\star}}
\newcommand{\typingRule}            {{\tt r}}
%\newcommand{\apvar}                     {{\ap_{\mbox{\tiny var}}}}
%\newcommand{\apnotvar}               {{\ap_{\neg\mbox{\tiny var}}}}
\newcommand{\apvar}                    {{{\tt ap}_{\mbox{\tiny $v$}}}}
\newcommand{\apnotvar}               {{{\tt ap}_{\mbox{\tiny $\neg v$}}}}
\newcommand{\subseteqsim}         {{\raisebox{2pt}{$\subset$}_{\!\!\!\!\!\sim}}}
\newcommand{\supseteqsim}         {{\raisebox{2pt}{$\supset$}_{\!\!\!\!\!\sim}}}
\newcommand{\Ctxt}                       {{\tt Ctxt}}
\newcommand{\Seq}                        {{\tt Seq}}
\newcommand{\Stat}                       {{\tt Stat}}
\newcommand{\Rule}                       {{\tt Rule}}
\newcommand{\restr}                      {{\lceil}}
\newcommand{\setT}                       {{\mathcal T}}
\newcommand{\base}                      {{\tt base}}
\newcommand{\inductive}               {{\tt ind}}
\newcommand{\universe}     [1]       {{|#1|}}
\newcommand{\Label}                     {{\tt Lbl}}
\newcommand{\quotationMarks} [1]     {{\textquotedblleft#1\textquotedblright}}
\newcommand{\lexicographic} [1]   {{ <^{#1}_{\mbox{\tiny lex}}} }
\newcommand{\trunk}                     {{\tt trunk}}






\ifdraft

\title{CT-$\lambda$, a cyclic simply typed $\lambda$-calculus with fixed points}

\author{Stefano Berardi }
\date{}

\else

\title[Equivalence]
{
???
}

\author[S. Berardi]{Stefano Berardi}
\address{Universit\`{a} di Torino,
Torino, Italy}
\email{stefano@unito.it}


%% mandatory lists of keywords 
\keywords{
proof theory
inductive definitions
Brotherston-Simpson conjecture
cyclic proofs
}
\fi

\maketitle

\begin{abstract}
We explore the possibility of defining a circular syntax
directly for $\lambda$-abstraction and simple types, instead of interpreting
 $\lambda$-abstraction through combinators first, then inserting a  circular syntax afterwards.
We introduce our circular syntax as a fixed point operator defined by cases.

We prove the expected results for this circular simply typed $\lambda$-calculus, which we call $\CTlambda$: 
\emph{(i)} every closed term of type $\N$ reduces to some numeral;
\emph{(ii)}  strong normalization for all reductions sequences outside all fixed points; 
\emph{(iii)}  strong normalization for ``fair'' reductions; \emph{(iv)} Church-Rosser property;
and \emph{(i)} the equivalence between circular syntax and ordinary syntax 
for G\"{o}del system $\systemT$. 

Our motivation is that terms of $\CTlambda$ 
are much shorter than the equivalent terms written in $\systemT$.
Beside, the circular syntax of $\CTlambda$, using binders instead of combinators, 
could be more familiar for researchers working in the field of Type Theory.

\end{abstract}

\iffalse
key words: 
proof theory,
inductive definitions,
Brotherston-Simpson conjecture,
cyclic proofs,
Martin-Lof's system of inductive definitions,
infinite Ramsey theorem
Podelski-Rybalchenko termination theorem
size-change termination theorem
\fi


%%%%%%%%%%%%%%%%%%%%%%%%%%%%%%%%%%%%%%%%%%%%%%%%%%%%%%
%AGGIUNGO TEOREMA LEMMA PROPOSIZIONE COROLLARIO PROVA% 
%%%%%%%%%%%%%%%%%%%%%%%%%%%%%%%%%%%%%%%%%%%%%%%%%%%%%%





\section{Definition of $\CTlambda$}

We define pseudo-terms of $\CTlambda$: the well-formed terms of $\CTlambda$ will be a proper
subset.

\begin{definition}[pseudo-terms $\CTlambda$]
\mbox{}
\begin{enumerate}

\item
The types of $\CTlambda$ are all types we can define with $\N$ and $\rightarrow$.

\item
The pseudo-terms of $\CTlambda$ 
are all possibly infinite trees we can define with $x^T$ ($T$ type), $\lambda x^T.t$, 
$\ap(t,u)$, $0$, $S(t)$ (successor of $t$), $\cond x.(f,g)$, $\nameIn(f,x)$ with $x \not \in \FV(f)$.

\item
The scope of the binder $\lambda x^T.t$ is $t$.
The scope of the binder $\cond x.(f,g)$ is $g$ ($f$ is \emph{not} in the scope of $\cond x.(f,g)$).

\end{enumerate}
When $t = S^n(0)$ we say that $t$ is a numeral.
\end{definition}

As an example we have the term $t = \cond x. (0,(\lambda y.t)(0))$. We say that a tree is regular if
it has finitely many subtrees. $t$ is an infinite regular tree,
with subtrees $t$, $0$, $(\lambda y.t)(0)$, $\lambda y.t$.

The typing rules are the usual ones but for $\cond$ and $\nameIn$. 

Assume $\Gamma$, $\Delta$ are sequents of length $n$, $m$respectively. Suppose
$f:\{1, \ldots, n\} \rightarrow \Delta$ is any injection.

\begin{enumerate}
\item
Structural rule $\struct(f)$.
If $t: \Gamma \vdash T$ then $t:\Delta \vdash T$

\item
If $t:\Gamma \vdash T$ and $\Gamma'$ is a renaming of $\Gamma$ then
$t:\Gamma' \vdash T$

\item
If $x \not \in \Gamma$ and $f:\Gamma \vdash\N \rightarrow T$
then $f(x) : \Gamma, x:\N \vdash T$.

\item
If $x:A \in \Gamma$ then $x:\Gamma \vdash A$.

\item
If $\Gamma \vdash f:A \rightarrow B$ and $a:\Gamma \vdash A$ then $f(a) : \Gamma \vdash B$.

\item
If $\Gamma, x:A \vdash b: B$ then $\lambda x:A.b : \Gamma \vdash A \rightarrow B$.

\item
$0:\Gamma \vdash \N$

\item
If $t:\Gamma \vdash \N$ then $S(t):\Gamma \vdash \N$.

\item
$\cond$. If $ f :\Gamma, x:T \vdash T$, $x \not \in \FV(f)$ and  $g : \Gamma, x:T \vdash T$
then $\cond x.(f,g) : \Gamma \vdash \N \rightarrow T$.
\end{enumerate}

$\nameIn$ is a particular case of $\ap$, when we replace $x:\N$ with some 
$a:\Gamma \vdash \N$  then we replace $\nameIn$ with $\ap$.

As an example if $t = \cond x. (0,(\lambda x.t)(0))$ is as above, then $t:\N \rightarrow \N$.  

Our goal is to provide a set of well-formed term for $\CTlambda$ and interpret them as total functionals. 
Our first step is to provide reduction rules for $\CTlambda$.


\begin{definition}[reduction rules for $\CTlambda$]
\mbox{}
\begin{enumerate}
\item
$\beta$: $(\lambda x^A.b)(a) \reduces_\beta b[a/x]$ and $\cond x.(f,g)(0) \reduces_\cond f$,
$\cond x.(f,g)(S(t)) \reduces_\cond g[t/x]$.

\item
$\reduces$ is the context and transitive closure of $\reduces_\beta$ and $\reduces_\cond$,
in which we forbid reductions in proper subterms of any $\cond x.(f,g)$. 

\end{enumerate}

When $t \reduces u$ we say that $t$ \emph{safely reduces} to $u$.
We call \emph{unsafe} a reduction inside any $\cond x.(f,g)$.
\end{definition}

As an example if $t = \cond x. (0, (\lambda x.t)(0) )$ is as above, then $t$ is normal, because
all redexes in $t$ are of the form  $(\lambda x.t)(0)$ and inside a $\cond$. 

The reason for forbiding
reductions inside $\cond x.(f,g)$ is that thorugh $\cond$ we will-express fixed-point equations.
Therefore reductions for $\cond$ can easily loop and they are ``unsafe". For this reason, we allow 
the minimum possible reductions concerning $\cond x.(f,g)$, those of the form
$\cond x.(f,g)(0) \reduces_\cond f$ and
$\cond x.(f,g)(S(t)) \reduces_\cond g[t/x]$ for maximal $\cond$-expression, and no reduction
inside the arguments $f$, $g$ of $\cond$.

As an example, if $n$ is any numeral, then 
$t(n) = \cond x. (0, (\lambda x.t)(0) )(2) : \N$ has infinitely many $\beta$-redexes inside 
$\cond$, therefore infinitely many unsafe reductions are possible. There are only finitely many safe reduction
from $t(n)$ instead: $t(n)$ reduces to to $t(n-1)$ for $n$ times, until we get $t(0)$, then $0$ and we stop. 

Indeed, we have $t(0) = \cond x. (0, (\lambda x.t)(0) )(0) = 0$, therefore 
$t(1) = \cond x. (0, (\lambda x.t)(0) )(1) = (\lambda x.t)(0)(0) = t(0) = 0$,
then $t(2) =  \cond x. (0, (\lambda x.t)(0) )(2) = (\lambda x.t)(0)(1) = t(1) = 0$, and so forth.

We define the well-formed terms of $\CTlambda$: for them we will prove
strong normalization, church-rosser for terms of type $\N$, and the fact that every term of type
$\N$ is a numeral. 
As a consequence, well-formed terms $\CTlambda$ will be interpreted as total functionals. 

The first step in defining well-formed terms is defining a correspondence between atoms in the
proof that $t$ is well-typed. We need first the notion of list of types and atomic types for a term.

\begin{definition}[Integer maps]
\begin{enumerate}
\item
The type list of $t$ is $\vec{C} = \vec{A},\vec{B}$. 

\item
An atom index of $t$ is any $j \in \{1, \ldots, n+m\}$ such that $C_j = \N$.

\item
We define $\ins(a,x)=x+1$ if $x \ge a+1$, and $\ins(a,x)=x$ if $x\ge a$.

\end{enumerate}
\end{definition}

We now define the atom correspondence in $\CTlambda$.



\begin{definition}[Atom correspondence in  $\CTlambda$]
Assume $\vec{A} = A_1, \ldots, A_n$, $\vec{B}=B_1, \ldots, B_m$
and $t:\vec{x}:\vec{A} \vdash \vec{B} \rightarrow \N$ is a pseudo-term.

\begin{enumerate}
\item
The type list of $t$ is $\vec{C} = \vec{A},\vec{B}$. 
An atom index of $t$ is any $j \in \{1, \ldots, n+m\}$ such that $C_j = \N$.

\item
For each atom index $k$ in $t$, each immediate subterms $t'$ of $t$ 
each atom index $k'$ in $t'$ we define the relation: ``$k',t'$ the successor of $k,t$". We require:
\begin{enumerate}
\item
if $t$ is obtained by a rule $\struct(f)$ then $k = f(k')$  if $k' \le n$ and $k = k' - n + m$ if $k \ge n+1$.
\item
$k' = \ins(n,k)$ if $t=f(a)$, $a:\N$ and $t'=f$. 
\item
$k = k'$ in all other cases.
% in all other cases: $S$, $\lambda$, $\nameIn$, $\cond$, and $\ap$ if different from the previous case.
\end{enumerate}
\end{enumerate}
\end{definition}

From the atom correspondence we define the global trace condition and the notion of well-formed term.

\begin{definition}[Global trace condition and well-formed terms in  $\CTlambda$]
\begin{enumerate}
\item
A path $\pi$ in $t$ is any list $t_0, \ldots, t_n$ of subterms of $t$ 
such that $t_0 = t$ and each $t_{i+1}$ is an immediate subterm of $t_i$. 

\item
Assume $\pi =t_0, \ldots, t_n$ is a branch of $t$. A trace of $\pi$ consists of an integer list 
$k_m, \ldots, k_n$ such that for each $i=m,\ldots, n$:
$k_i$ is an atom index of $t_i$, and if $i+1 \le n$ then $k_{i+1},t_{i+1}$ is the successor of $k_i, t_i$.

\item
Assume $\tau = k_m, \ldots, k_n$ is a trace of $\pi =t_0, \ldots, t_n$ and $i=m,\ldots, n$.
$\tau$ is progressing in $i$ if $t_i=\cond x.(f,g)$ for some $f$, $g$,
and $k_i$ is the index of the first argument the $\cond$-rule, otherwise $\tau$ is not progressing in $i$.

\item
$t$ satisfies the global trace condition if for all infinite paths $\pi$ in $t$ 
there is some infinitely progressing path $\tau$ in $\pi$.

\item
Well-formed terms are all terms which are regular trees (having finitely many subtrees), 
and satisfying the global trace condition.

\end{enumerate}
\end{definition}


\section{Examples of terms of $\CTlambda$}

\subsection{The sum map}
As a first example of term of  $\CTlambda$ we can provide a well-formed term computing the sum on $\N$.
\ldots\ldots\ldots\ldots\ldots\ldots\ldots\ldots\ldots\ldots\ldots


\subsection{The iterator}
A second example. We define a term $\It$ of  $\CTlambda$ computing the iteration of maps on $\N$.
We define a normal term $\It:(\N \rightarrow \N), \N,\N \rightarrow \N$ such that
$\It(f,n,a)=f^n(a)$ for all $n \in \N$. 
We have to to solve the equations $\It(f,a,0) = a$ and $\It(f,a,S(t)) = f(\It(f,a,t))$.
We solve them with $\It = \lambda f, a,x. i[a,f,x]$
with $ i[a,f,x] = (\cond x. (a, f( i[a,f,x]) ))(x)$.

The term is well-typed and regular by definition. We check the global trace condition. 
We mark the last unnamed argument $\N$ of $\It$. The mark moves to $x$ in $ i[a,f,x]$.
 Through a $\nameIn$-rule the mark moves to the unique unnamed argument $\N$ of  
$ \cond x. (a, f( i[a,f,x]) )$.
The mark either moves to the name $x$ in the context of $a$ and there it stops, or 
it progresses and moves to $x$ in the context of $f(i[f,a,x])$, then in the context of $i[f,a,x]:\N$.
%??????????????? 10:30 24/03/24
Now we loop: if the path continues forever, then after infinitely many step the mark from which we started 
progresses infinitely many times, each time it crosses a $\cond$-rule.



\subsection{The Interval Map}
A third example. We simulate interval with two variables $\nil:\alpha$ and 
$\cons:\N,\alpha \rightarrow \alpha$. We recursively define a notation for maps by $[]=\nil$
and $[a,\vec{a}] = \cons(a,[\vec{a}])$. We add no elimination rules for lists, though,
only the variables $\nil$ and $\cons$.

We will define a term $\Interval$ with one argument $f:\N \rightarrow \N)$ and three argument
$a,x,y:\N$, such that 
$$
\Interval(f,a,n,m) = [f^n(a), f^{n+1}(a), \ldots, f^{n+m}(a)]
$$ 
for all $n,m \in \N$. 
We have to to solve the recursive equations 
$$
\Interval(f,a,x,0) = \cons(f^n(a), \nil) 
\ \ \ \ \ \ 
\Interval(f,a,x,S(t)) = \cons(f^n(a), \Interval(f,a,x,t) )
$$ 
We solve them with $\Interval = \lambda f,a.v$,
where $v:\N,\N\rightarrow \N$, 
$v = \lambda x.\cond y. (\ \cons(w, \nil),  \  \cons(w, v(S(x),y)) \ )$ and $w=\It(f,a,x)$.

The term is well-typed and regular by definition. We check the global trace condition.
We mark the last unnamed argument $\N$ of $\Interval$.
The mark moves to the last unnamed argument $\N$ of  
$v:\N,\N \rightarrow \N$. 
We unfold $v$ to $\lambda x.\cond y. 
(\ \cons(w, \nil),  \  \cons(w, v(S(x),y)) \ )$
The mark progresses and moves to the name $y$ in the context of the body of the $\lambda$-abstraction,
then to $y$ in the context of $\cons(w, \nil)$ or of $\cons(w, v(S(x),y))$,
then in the context of $\nil$ and stops, or in the context of $w=\It(f,a,x):\N$, 
for which we already checked the global trace condition, or in the context of $v(S(x),y)):\alpha$. 

Then the mark moves from the named argument $y$ of $v(S(x)):\N \rightarrow \N$ to the unique 
unnamed argument $\N$ of $v(S(x))$, and eventually to the last argument of $v:\N,\N\rightarrow \N$. 
From $v$ we loop: after infinitely many step either we reached some $w$ and we find 
some infinite progressing trace inside it, or the
mark from which we started progresses infinitely many times through $v$. In both cases we have the
global trace condition.

We have infinitely many nested $\beta$-reduction $(\lambda x. \ldots)(S(x))$.
We can remove all of them in a single step. Inside the $\beta$-redex number $n$ we obtain a sub-term
$w[S(x)/x]\ldots[S(x)/x]$ (substitution repeated $n$ times).
The result is $w[S^n(x)] = \It(f,a,S^n(x))$.
The nested substitution produce new $\beta$-reductions 
$\It(f,a,S^n(x)) = (\lambda x.i[f,a,x])(S^n(x))$ for all $n \in \N$.
This is a non-regular term: we have infinitely many pairwise different 
sub-terms $x,S(x),S(S(x)), S(S(S(x))), \ldots$.
We need infinitely many steps to normalize all $\It(f,a,S^n(x))$ to $f^n(i[f,a,x])$, 
even if we allow to reduce all $\beta,\cond$-redexes at the same time.
Also the normal form is not regular: it contains all terms $f^n(i[f,a,x])$ for $n \in \N$, hence
infinitely many pairwise different terms. These infinite sub-terms are of a particulary simple form, though. 
They are obtained by the repeating $n$ times the assignment $z:=f(z)$, then applying $z:=i[f,a,x]$ once
to the result.

Apparently, $\Interval$ 
is some term of $\CTlambda$ which cannot be normalized in finite time, not even if we allow
infinite parallel reductions without any "safety" restriction. The normal form is produced \emph{only in the limit}
and it is \emph{not regular}. If we allow to reduce infinitely many nested $\beta$-redexes in one step, also
the intermediate steps of the infinite reduction of $\Interval$ are not regular.


\section{Weak normalization for closed terms of$\CTlambda$ of type $\N$}
\label{section-weak-normalization}
In this section we prove that every closed term of$\CTlambda$
 (that is, well-typed, regular and with the global trace condition) normalizes with finitely many "safe" steps
to some numeral $n \in \N$.

\ldots\ldots\ldots\ldots\ldots\ldots\ldots\ldots\ldots\ldots

\section{Uniqueness of normal form for closed terms of $\CTlambda$ of type $\N$}
In this section we prove a weak form of confluence: normal form for
all closed terms of $\CTlambda$ of type $\N$ is unique.

By the weak normalization result proved in \S \ref{section-weak-normalization},
we will deduce: for all for all closed terms $t$ of $\CTlambda$ of type $\N$, 
there is some $n \in \N$ such that there is a safe reduction $t \reduces n$, and all normal form of $t$
are equal to $n$.

We define a notion $\sim$ of equivalence for well-typed terms with the same type.
Assume $t, u : A$ are \emph{closed} terms of $\CTlambda$.
 (that is, well-typed, regular and with the global trace condition). 

We define an equivalence $\sim_0$ for closed terms,
by induction on the type $A$ of $t$, $u$.

\begin{enumerate}
\item
Assume $A = \N$. Then $t \sim_0 u$ if and only if for all normal $t',u':\N$, if $t \reduces t'$
and $u \reduces u'$ then $t'=u'$.
\item
Assume $A = B \rightarrow C$. Then $t \sim_0 u$ if and only if for all closed $b,c:B$ if
$b \sim_0 c$ then $t(b) \sim_0 u(c)$.
\end{enumerate}

We will prove that for all closed terms $t$ of $\CTlambda$ we have $t \sim_0 t$.
If $A=\N$ this means that the normal form of all closed terms of $\CTlambda$ of type $\N$ is unique,
which is our goal.

We first define $t \sim u$ for \emph{any} terms of $\CTlambda$ with the same type.
 $t \sim u$ if and only if for any two substitution 
$\sigma = [\vec{t} / \vec{x}]$ and $\tau = [\vec{u} / \vec{x}]$,
if $\vec{t} \sim_0 \vec{u}$ and $\sigma(t)$, $\sigma(u)$ are closed then $\sigma(t) \sim_0 \sigma(u)$.

We will prove that  for all terms $t$ of $\CTlambda$ we have $t \sim t$. In the case of closed terms $t$
we can choose $\sigma=\tau=$ the empty sbustitution and we obtain $t \sim_0 t$, which is our goal.

Assume that $t:\N$ is a closed term $t$ of $\CTlambda$ and $n \in \N$. 
\begin{enumerate}
\item
We say that $n$ is a value of $t$ if $t \reduces n$ (if $t$ safely reduces to $n$). 
\item
We say that $t$ is confluent if $t \sim_0 t$.
\end{enumerate}
Assume $\vec{x}:\vec{A}$ and $t[\vec{x}]: \vec{D} \rightarrow \N$. Assume 
$\vec{a}:\vec{A},\vec{d}:\vec{D}$ 
and $\vec{b}:\vec{A},\vec{e}:\vec{D}$ are vectors of closed terms of $\CTlambda$. We say that 
these four vectors are a counterexample to $t$ if $\vec{a},\vec{d} \sim_0 \vec{b},\vec{e}$
and $\neg (t[\vec{a}](\vec{d}) \sim_0  t[\vec{b}](\vec{e}))$.

We have $t \sim t$ if and only if there is no counter-example for $t$.


By the weak normalization result proved in \S \ref{section-weak-normalization}, all 
closed term $t$ of $\CTlambda$ of type $\N$ have a value. If $t$ is confluent, then the value is unique.
For confluent closed term $t$ of $\CTlambda$ of type $\N$, "safe" reduction preserves the value. Indeed,
if $t \reduces u$ then $u$ is a closed term of $\CTlambda$ of type $\N$, therefore $u$ has a value
$m$, which is also a value of $t$. By confluence of $t$ we conclude that $n=m$.

If $t$ is as above, with value $n$, and $t \reduces S(u)$, then $n>0$, 
$u$ is a confluent closed term $t$ of $\CTlambda$ of type $\N$, and the value of $u$ is $n-1$.
Indeed,if $u \reduces u'$ and $u'$ is normal, then $t \reduces S(u')$, therefore $S(u')=n$ 
by confluence of $t$, and we conclude that $n>0$ and $u'=n-1$. Thus, the normal form of $u$ is unique.

We will prove the following.

\begin{theorem}
Assume $t$ is any well-typed term such that $\neg (t \sim t)$. Then there is some infinite path 
$\pi=\{t_i | i \in \N\}$ of $t$ with some infinite sequence $\sigma= \{\sigma_i| i \in \N\}$, with
each $\sigma_i$ counter-example for $t_i$, and with the value of $\sigma$ compatible with the
trace conditions of $\pi$.
\end{theorem}

As a corollary from the theorem we will conclude: 
if $t$ satisfies the global trace condition, then there is no such $\sigma$, therefore $t \sim t$, 
as we wished to show.

We prove that for any term $ (t \sim t)$ with counter-example $\vec{a},\vec{d} \sim_0 \vec{b},\vec{e}$
and $\neg (t[\vec{a}](\vec{d}) \sim_0  t[\vec{b}](\vec{e}))$ we can find some immediate
subterm $t'$ and with a counter-example $\vec{a'},\vec{d'} \sim_0 \vec{b'},\vec{e'}$
and $\neg (t'[\vec{a'}](\vec{d'}) \sim_0  t'[\vec{b'}](\vec{e'}))$, \emph{compatible with trace
condition}.


%21:09 20/03/2024

%
%\[
%	\infer[(\Case\ E)]{Ex \prove \N x}{
%		\infer[(=L)]{x=0 \prove \N x}{
%		\infer[(Wk)]{x=0 \prove \N 0}{
%		\infer[(\N\ R)]{\prove \N 0}{}
%		}}
%		&
%		\infer[(=L)]{x=sx',Ox' \prove \N x}{
%		\infer[(\N\ R)]{x=sx',Ox' \prove \N sx'}{
%		\infer[(Wk)]{x=sx',Ox' \prove \N x'}{
%		\infer[(\Subst)]{Ox' \prove \N x'}{
%		Ox \prove \N x
%		}}}}
%	}
%\]
%

\section{appendix}

\begin{verbatim}

To: kmr@is.sci.toho-u.ac.jp (Daisuke Kimura)
Re: proof of Weak Normalization to an integer for CT-lambda
Fri, 22 Mar 2024 08:25:57 +0100 

    By the way, I re-checked the weak curry-howard proof, now i think that the proof 
does not require the property p-->q, a-->b ==> p[a/x]-->q[b/x] and can be completed 
with the notion of safe reduction.
but in fact it would be more interesting to prove full church-rosser for Circular T-lambda, 
as anupam does for his circular T.

    About strong normalization, we can prove it for "safe" reductions, those inside no cond. 
More in general, we know that we can have infinite reduction sequences, because we can 
have infinitely many redexes. However, for any infinite reduction sequence sigma, I conjecture 
we can prove a kind of stabilization of the term. After some reduction step, the term only 
changes inside some cond nested k times. 

    Namely, I conjecture that

"for any cyclic lambda term t, any infinite reduction sequence (sigma(n)|n in N) with sigma(0)=t, 
any k in N, there is a n0 in N such that for all n>=n0, the terms sigma(n) and sigma(n0)  
coincide on all branches with at most k times cond."

    Best, Stefano

\end{verbatim}


We say that a node is in the $k,\cond$-level if it has less than $k$ nodes $\cond$ in its branch.

We say that an infinite reduction sequence is is fair for nodes of $k,\cond$-level in the following case:
for all $i\in \N$, all $k>0$, there is some  $j \ge i$ such that at step $j$ either all nodes in the 
$k,\cond$-level are normal, or one of them is reduced at step $j$.

\emph{Conjecture}. An infinite reduction sequence $\sigma$ 
is normalizing in the limit if and only if for all $k \in \N$,
$k > 0$, $\sigma$ is fair for the task of reducing nodes in the $k,\cond$-level.



\end{document}