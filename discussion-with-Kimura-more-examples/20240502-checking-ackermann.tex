\documentclass{article}

\usepackage{amssymb,amsmath,proof,graphicx,color}

\newcommand{\bfColor}[2]{{\bf \color{#1}{#2}}}
\definecolor{oldgold}{rgb}{0.81, 0.71, 0.23}

\newcommand{\Rapv}{\hbox{$(\text{ap}_{\text{v}})$}}
\newcommand{\RapNv}{\hbox{$(\text{ap}_{\lnot\text{v}})$}}
\newcommand{\Reta}{\hbox{$(\eta)$}}
\newcommand{\Rap}{\hbox{$(\text{ap})$}}
\newcommand{\Rcond}{\hbox{$(\text{cond})$}}
\newcommand{\FV}[1]{\text{FV}(#1)}
\newcommand{\Ack}{{\tt ack}}
\newcommand{\AckA}{{\tt ack1}}
\newcommand{\AckB}{{\tt ack2}}
\newcommand{\ACK}{{\rm Ack}}
\newcommand{\Cond}[2]{\text{cond}(#1,#2)}
\newcommand{\Suc}[1]{{\tt S}#1}
\newcommand{\N}{N}
\newcommand{\gN}{\bfColor{oldgold}{N}}
\newcommand{\rN}{\bfColor{red}{N}}
\newcommand{\rM}{\bfColor{red}{m}}
\newcommand{\bN}{\bfColor{blue}{n}}

\newtheorem{theorem}{Theorem}[section]
\newtheorem{lemma}[theorem]{Lemma}
\newtheorem{proposition}[theorem]{Proposition}
\newtheorem{corollary}[theorem]{Corollary}
\newtheorem{definition}[theorem]{Definition}

\newenvironment{proof}[1][Proof]{\begin{trivlist}
\item[\hskip \labelsep {\bfseries #1}]}{\end{trivlist}}
\newenvironment{example}[1][Example]{\begin{trivlist}
\item[\hskip \labelsep {\bfseries #1}]}{\end{trivlist}}
\newenvironment{remark}[1][Remark]{\begin{trivlist}
\item[\hskip \labelsep {\bfseries #1}]}{\end{trivlist}}
\newenvironment{claim}[1][Claim]{\begin{trivlist}
\item[\hskip \labelsep {\bfseries #1}]}{\end{trivlist}}


\begin{document}

{\bf On Ackermann function and inference rules $\Reta$ and $\Rapv$}

\begin{align*}
  \ACK(0,n) &= n+1
  \\
  \ACK(m+1,0) &= \ACK(m,1)
  \\
  \ACK(m+1,n+1) &= \ACK(m,\ACK(m+1,n))
\end{align*}

Remark that this is defined by the lexicographic order of $(m,n)$. 

\section{First term representation: $\AckA$}

We first give the first representation $\AckA$ of Ackermann function. 
In this article, we sometimes write $f(x,y)$ instead of $f(x)(y)$, for readability.

\begin{definition}[$\AckA$]\rm
  We define $\AckA$ by the following equation.
  \[
  \AckA = \lambda \rM\bN.\Cond{\Suc{\bN}}{\lambda m'.\Cond{\AckA(m',1)}{\lambda n'.\AckA(m',\AckA(\Suc{m'},n'))}(\bN)}(\rM)
  \]
\end{definition}

\noindent{\bf Checking the behavior of $\AckA$}
\begin{align*}
  \AckA(0,n)
  &\mapsto
  \Cond{\Suc{n}}{\lambda m'.\Cond{\AckA(m',1)}{\lambda n'.\AckA(m',\AckA(\Suc{m'},n'))}(n)}(0)
  \\
  &\mapsto
  \Suc{n}
\end{align*}
%%%%%%%
\begin{align*}
  \AckA(\Suc{m},0)
  &\mapsto
  \Cond{\Suc{0}}{\lambda m'.\Cond{\AckA(m',1)}{\lambda n'.\AckA(m',\AckA(\Suc{m'},n'))}(0)}(\Suc{m})
  \\
  &\mapsto
  (\lambda m'.\Cond{\AckA(m',1)}{\lambda n'.\AckA(m',\AckA(\Suc{m'},n'))}(0))(m)
  \\
  &\mapsto
  \Cond{\AckA(m,1)}{\lambda n'.\AckA(m',\AckA(\Suc{m'},n'))}(0)
  \\
  &\mapsto
  \AckA(m,1)
\end{align*}
%%%%%%%
\begin{align*}
  \AckA(\underline{\Suc{m}},\Suc{n})
  &\mapsto
  \Cond{\Suc{\Suc{n}}}{\lambda m'.\Cond{\AckA(m',1)}{\lambda n'.\AckA(m',\AckA(\Suc{m'},n'))}(\Suc{n})}(\underline{\Suc{m}})
  \\
  &\mapsto
  (\lambda m'.\Cond{\AckA(m',1)}{\lambda n'.\AckA(m',\AckA(\Suc{m'},n'))}(\Suc{n}))(\underline{m})
  \\
  &\mapsto
  \Cond{\AckA(m,1)}{\lambda n'.\AckA(m,\AckA(\Suc{\underline{m}},n'))}(\Suc{n})
  \\
  &\mapsto
  (\lambda n'.\AckA(m,\AckA(\Suc{\underline{m}},n')))(n)
  \\
  &\mapsto
  \AckA(m,\AckA(\Suc{\underline{m}},n))
\end{align*}

The point of $\AckA$ is the last case. The term $\underline{\Suc{m}}$ of the first line and $\Suc{\underline{m}}$ of the last line are slightly diffelent: $\underline{\Suc{m}}$ is decomposed into $\underline{m}$ by the $\text{cond}$-reduction, then $\Suc{\underline{m}}$ is constructed by substituting $m'$ of $\Suc{m'}$ by $\underline{m}$. 

\subsection{$\AckA:\N\to\N\to\N$ is not in GTC (with $\Reta$)}

\begin{claim}
  $\AckA:\N\to\N\to\N$ is well-typed, but does not satisfy GTC in the system with $\Reta$.
\end{claim}

{\footnotesize
  \hspace{-3cm}
  $\infer{
    \vdash \AckA:\rN\to\gN\to\N\ (\dagger)
  }{
    \infer[\Reta]{
      m:\rN, n:\gN \vdash \Cond{\Suc{n}}{\lambda m'.\Cond{\AckA(m',1)}{\lambda n'.\AckA(m',\AckA(\Suc{m'},n'))}(n)}(m): \N
    }{
      \infer[\Rcond]{
        n:\gN \vdash \Cond{\Suc{n}}{\lambda m'.\Cond{\AckA(m',1)}{\lambda n'.\AckA(m',\AckA(\Suc{m'},n'))}(n)}: \rN\to\N
      }{
        \infer{
          n:\N \vdash \Suc{n}: \N
        }{
          \infer{
            n:\N \vdash n: \N
          }{}
        }
        &
        \infer{
          n:\gN \vdash \lambda m'.\Cond{\AckA(m',1)}{\lambda n'.\AckA(m',\AckA(\Suc{m'},n'))}(n): \rN\to\N
        }{
          \infer[\Reta]{
            n:\gN, m':\rN \vdash \Cond{\AckA(m',1)}{\lambda n'.\AckA(m',\AckA(\Suc{m'},n'))}(n): \N
          }{
            \infer[\Rcond]{
              m':\rN \vdash \Cond{\AckA(m',1)}{\lambda n'.\AckA(m',\AckA(\Suc{m'},n'))}: \gN\to\N
            }{
              \infer{
                m':\rN \vdash \AckA(m',1):\N
              }{
                \infer[\Reta]{
                  m':\rN \vdash \AckA(m'):\N\to\N
                }{
                  \deduce{
                    \vdash \AckA:\rN\to\N\to\N
                  }{(\dagger1)}
                }
                &
                \infer{
                  \vdash 1:\N
                }{}
              }
              &
              \infer{
                m':\rN \vdash \lambda n'.\AckA(m',\AckA(\Suc{m'},n')): \gN\to\N
              }{
                \infer{
                  m':\rN, n':\gN \vdash \AckA(m',\AckA(\Suc{m'},n')): \N
                }{
                  \infer[\Reta]{
                    m':\rN \vdash \AckA(m'): \gN\to\N
                  }{
                    \deduce{
                      \vdash \AckA: \rN\to\gN\to\N
                    }{(\dagger2)}
                  }
                  &
                  \infer[\Reta]{
                    m':\rN, n':\gN \vdash \AckA(\Suc{m'},n'): \N
                  }{
                    \infer[(*)]{
                      m':\rN \vdash \AckA(\Suc{m'}): \gN\to\N
                    }{
                      \deduce{
                        \vdash \AckA:\N\to\gN\to\N
                      }{
                        (\dagger3)
                      }
                      &
                      \infer{
                        m':\rN \vdash \Suc{m'}:\N
                      }{
                        \infer{
                          m':\rN \vdash m':\N
                        }{}
                      }
                    }
                  }
                }
              }
            }
          }
        }
      }
    }
  }$  
}

The problem is $\Rap$.
This cuts the trace chasing of $\rN$. 
The infinite path $(\dagger)\rightsquigarrow(\dagger1)\rightsquigarrow(\dagger)\rightsquigarrow(\dagger3)\rightsquigarrow(\dagger)\rightsquigarrow(\dagger1)\rightsquigarrow(\dagger)\rightsquigarrow(\dagger3)\rightsquigarrow\cdots$ does not contains progressing trace. 

\subsection{$\AckA:\N\to\N\to\N$ is not in GTC (with $\Rapv$)}

\begin{claim}
  $\AckA:\N\to\N\to\N$ is well-typed, but does not satisfy GTC in the system with $\Rapv$.
\end{claim}

The situation is the same as before. In the following, weakening is implicitly applied. 

{\scriptsize
  \hspace{-3cm}
  $\infer{
    \vdash \AckA:\rN\to\gN\to\N\ (\dagger)
  }{
    \infer[\Rapv]{
      m:\rN, n:\gN \vdash \Cond{\Suc{n}}{\lambda m'.\Cond{\AckA(m',1)}{\lambda n'.\AckA(m',\AckA(\Suc{m'},n'))}(n)}(m): \N
    }{
      \infer[\Rcond]{
        m:\N, n:\gN \vdash \Cond{\Suc{n}}{\lambda m'.\Cond{\AckA(m',1)}{\lambda n'.\AckA(m',\AckA(\Suc{m'},n'))}(n)}: \rN\to\N
      }{
        \infer{
          n:\N \vdash \Suc{n}: \N
        }{
          \infer{
            n:\N \vdash n: \N
          }{}
        }
        &
        \infer{
          n:\gN \vdash \lambda m'.\Cond{\AckA(m',1)}{\lambda n'.\AckA(m',\AckA(\Suc{m'},n'))}(n): \rN\to\N
        }{
          \infer[\Rapv]{
            n:\gN, m':\rN \vdash \Cond{\AckA(m',1)}{\lambda n'.\AckA(m',\AckA(\Suc{m'},n'))}(n): \N
          }{
            \infer[\Rcond]{
              n:\N, m':\rN \vdash \Cond{\AckA(m',1)}{\lambda n'.\AckA(m',\AckA(\Suc{m'},n'))}: \gN\to\N
            }{
              \infer{
                m':\rN \vdash \AckA(m',1):\N
              }{
                \infer[\Rapv]{
                  m':\rN \vdash \AckA(m'):\N\to\N
                }{
                  \infer{
                    m':\N \vdash \AckA:\rN\to\N\to\N
                  }{
                    \deduce{
                      \vdash \AckA:\rN\to\N\to\N
                    }{(\dagger1)}
                  }
                }
                &
                \infer{
                  \vdash 1:\N
                }{}
              }
              &
              \infer{
                m':\rN \vdash \lambda n'.\AckA(m',\AckA(\Suc{m'},n')): \gN\to\N
              }{
                \infer{
                  m':\rN, n':\gN \vdash \AckA(m',\AckA(\Suc{m'},n')): \N
                }{
                  \infer[\Rapv]{
                    m':\rN \vdash \AckA(m'): \gN\to\N
                  }{
                    \infer{
                      m':\N \vdash \AckA: \rN\to\gN\to\N
                    }{
                      \deduce{
                        \vdash \AckA: \rN\to\gN\to\N
                      }{(\dagger2)}
                    }
                  }
                  &
                  \infer[\Rapv]{
                    m':\rN, n':\gN \vdash \AckA(\Suc{m'},n'): \N
                  }{
                    \infer[\RapNv]{
                      m':\rN, n':\N \vdash \AckA(\Suc{m'}): \gN\to\N
                    }{
                      \deduce{
                        \vdash \AckA:\N\to\gN\to\N
                      }{
                        (\dagger3)
                      }
                      &
                      \infer{
                        m':\rN \vdash \Suc{m'}:\N
                      }{
                        \infer{
                          m':\rN \vdash m':\N
                        }{}
                      }
                    }
                  }
                }
              }
            }
          }
        }
      }
    }
  }$  
}

\section{Second term representation: $\AckB$}

We give the second representation $\AckB$ of Ackermann function. 

\begin{definition}[$\AckB$]\rm
  We define $\AckB$ by the following equation.
  \[
  \AckB = \lambda \rM\bN.\Cond{\Suc{\bN}}{\lambda m'.\Cond{\AckB(m',1)}{\lambda n'.\AckB(m',\AckB(\rM,n'))}(\bN)}(\rM)
  \]
\end{definition}

\noindent{\bf Checking the behavior of $\AckB$}

We check only the last case. 
\begin{align*}
  \AckB(\underline{\Suc{m}},\Suc{n})
  &\mapsto
  \Cond{\Suc{\Suc{n}}}{\lambda m'.\Cond{\AckB(m',1)}{\lambda n'.\AckB(m',\AckB(\underline{\Suc{m}},n'))}(\Suc{n})}(\underline{\Suc{m}})
  \\
  &\mapsto
  (\lambda m'.\Cond{\AckB(m',1)}{\lambda n'.\AckB(m',\AckB(\underline{\Suc{m}},n'))}(\Suc{n}))(\underline{m})
  \\
  &\mapsto
  \Cond{\AckB(m,1)}{\lambda n'.\AckB(m,\AckB(\underline{\Suc{m}},n'))}(\Suc{n})
  \\
  &\mapsto
  (\lambda n'.\AckB(m,\AckB(\underline{\Suc{m}},n')))(n)
  \\
  &\mapsto
  \AckB(m,\AckB(\underline{\Suc{m}},n))
\end{align*}

The point of $\AckB$ is that $\underline{\Suc{m}}$ at the first line and the one at the last line are exactly the same.

\subsection{$\AckB:\N\to\N\to\N$ is not in GTC (with $\Reta$)}

\begin{claim}
  $\AckB:\N\to\N\to\N$ is well-typed, but does not satisfy GTC in the system with $\Reta$.
\end{claim}

{\scriptsize
  \hspace{-3cm}
  $\infer{
    \vdash \AckB:\rN\to\gN\to\N\ (\dagger)
  }{
    \infer[\Rap]{
      m:\rN, n:\gN \vdash \Cond{\Suc{n}}{\lambda m'.\Cond{\AckB(m',1)}{\lambda n'.\AckB(m',\AckB(m,n'))}(n)}(m): \N
    }{
      \infer[\Rcond]{
        m:\rN, n:\gN \vdash \Cond{\Suc{n}}{\lambda m'.\Cond{\AckB(m',1)}{\lambda n'.\AckB(m',\AckB(m,n'))}(n)}: \N\to\N
      }{
        \infer{
          n:\N \vdash \Suc{n}: \N
        }{
          \infer{
            n:\N \vdash n: \N
          }{}
        }
        &
        \infer{
          m:\rN, n:\gN \vdash \lambda m'.\Cond{\AckB(m',1)}{\lambda n'.\AckB(m',\AckB(m,n'))}(n): \N\to\N
        }{
          \infer[\Reta]{
            m:\rN, n:\gN, m':\N \vdash \Cond{\AckB(m',1)}{\lambda n'.\AckB(m',\AckB(m,n'))}(n): \N
          }{
            \infer[\Rcond]{
              m:\rN, m':\N \vdash \Cond{\AckB(m',1)}{\lambda n'.\AckB(m',\AckB(m,n'))}: \gN\to\N
            }{
              \infer{
                m':\N \vdash \AckB(m',1): \N
              }{
                \infer[\Reta]{
                  m':\N \vdash \AckB(m'): \N\to\N
                }{
                  \deduce{
                    \vdash \AckB: \N\to\N\to\N
                  }{(\dagger1)}
                }
                &
                \infer{
                  \vdash 1:\N
                }{}
              }
              &
              \infer{
                m:\rN, m':\N \vdash \lambda n'.\AckB(m',\AckB(m,n')): \gN\to\N
              }{
                \infer{
                  m:\rN, m':\N, n':\gN \vdash \AckB(m',\AckB(m,n')): \N
                }{
                  \infer[\Reta]{
                    m':\N \vdash \AckB(m'): \N\to\N
                  }{
                    \deduce{
                      \vdash \AckB: \N\to\N\to\N
                    }{(\dagger2)}
                  }
                  &
                  \infer[\Reta]{
                    m:\rN, n':\gN \vdash \AckB(m,n'): \N
                  }{
                    \infer[\Reta]{
                      m:\rN \vdash \AckB(m): \gN\to\N
                    }{
                      \deduce{
                        \vdash \AckB: \rN\to\gN\to\N
                      }{(\dagger3)}
                    }
                  }
                }
              }
            }
          }
        }
      }
      &
      \infer{
        m:\rN \vdash m:\N
      }{}
    }
  }$
}

The problem is $\Rap$.
This cuts the trace chasing of $\rN$. 
The infinite path $(\dagger)\rightsquigarrow(\dagger1)\rightsquigarrow(\dagger)\rightsquigarrow(\dagger1)\rightsquigarrow\cdots$ does not contains progressing trace. 

\subsection{$\AckB:\N\to\N\to\N$ is not in GTC (with $\Rapv$)}

\begin{claim}
  $\AckB:\N\to\N\to\N$ is well-typed, but does not satisfy GTC in the system with $\Rapv$.
\end{claim}

{\scriptsize
  \hspace{-3cm}
  $\infer{
    \vdash \AckB:\rN\to\gN\to\N\ (\dagger)
  }{
    \infer[\Rapv]{
      m:\rN, n:\gN \vdash \Cond{\Suc{n}}{\lambda m'.\Cond{\AckB(m',1)}{\lambda n'.\AckB(m',\AckB(m,n'))}(n)}(m): \N
    }{
      \infer[\Rcond]{
        m:\N, n:\gN \vdash \Cond{\Suc{n}}{\lambda m'.\Cond{\AckB(m',1)}{\lambda n'.\AckB(m',\AckB(m,n'))}(n)}: \rN\to\N
      }{
        \infer{
          n:\N \vdash \Suc{n}: \N
        }{
          \infer{
            n:\N \vdash n: \N
          }{}
        }
        &
        \infer{
          m:\N, n:\gN \vdash \lambda m'.\Cond{\AckB(m',1)}{\lambda n'.\AckB(m',\AckB(m,n'))}(n): \rN\to\N
        }{
          \infer[\Rapv]{
            m:\N, n:\gN, m':\rN \vdash \Cond{\AckB(m',1)}{\lambda n'.\AckB(m',\AckB(m,n'))}(n): \N
          }{
            \infer[\Rcond]{
              m:\N, n:\N, m':\rN \vdash \Cond{\AckB(m',1)}{\lambda n'.\AckB(m',\AckB(m,n'))}: \gN\to\N
            }{
              \infer{
                m':\rN \vdash \AckB(m',1): \N
              }{
                \infer[\Rapv]{
                  m':\rN \vdash \AckB(m'): \N\to\N
                }{
                  \deduce{
                    \vdash \AckB: \rN\to\N\to\N
                  }{(\dagger1)}
                }
                &
                \infer{
                  \vdash 1:\N
                }{}
              }
              &
              \infer{
                m:\N, m':\rN \vdash \lambda n'.\AckB(m',\AckB(m,n')): \gN\to\N
              }{
                \infer{
                  m:\N, m':\rN, n':\gN \vdash \AckB(m',\AckB(m,n')): \N
                }{
                  \infer[\Reta]{
                    m':\rN \vdash \AckB(m'): \N\to\N
                  }{
                    \deduce{
                      \vdash \AckB: \rN\to\N\to\N
                    }{(\dagger2)}
                  }
                  &
                  \infer[\Rapv]{
                    m:\N, n':\gN \vdash \AckB(m,n'): \N
                  }{
                    \infer[\Reta]{
                      m:\N \vdash \AckB(m): \gN\to\N
                    }{
                      \deduce{
                        \vdash \AckB: \N\to\gN\to\N
                      }{(\dagger3)}
                    }
                  }
                }
              }
            }
          }
        }
      }
    }
  }$
}

The problem is the lower most $\Rapv$.
This cuts the trace chasing of $m:\rN$. 
Perhaps $m:\N$ at $(\dagger3)$ should be $\rN$. 

\vspace{1cm}

Summing up, both $\Reta$ and $\Rapv$ are problematic.
Possible solution might be: 
\[
\infer[\Rapv]{
  \Gamma,x:\rN \vdash f(x):A
}{
  \Gamma,x:\rN \vdash f:\rN\to A
}
\]


  
\end{document}

\infer{}{}

\lambda mn.\Cond{\Suc{n}}{\lambda m'.\Cond{\Ack(m')(1)}{\lambda n'.\Ack(m')(\Ack(\Suc{m'})(n'))}(n)}(m)